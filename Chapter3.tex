\documentclass{article}
\usepackage{amsmath}
\usepackage{amsfonts}
\usepackage{amssymb}

\newenvironment{proof}{\paragraph{Proof:}}{\hfill$\square$}
\newtheorem{theorem}{Theorem}
\newtheorem{lemma}[theorem]{Lemma}
\newtheorem{corollary}[theorem]{Corollary}

\newcommand{\R}{\mathbb{R}}
\newcommand{\Q}{\mathbb{Q}}
\newcommand{\N}{\mathbb{N}}


\newcommand{\B}{\mathbb{B}}
\newcommand{\osc}{\text{osc}}

\author{Arthur Chen}
\title{Chapter 3 Functions of a Real Variable}
\date{\today}

\begin{document}
\maketitle
\section*{Problem 1}
Assume that $f: \R \rightarrow \R$ satisfies $|f(t) - f(x)| \leq |t-x|^2$ for all $t, x$. Prove that $f$ is constant.

\begin{proof}
The assumption implies that for all $t, x$,
\[
0 \leq \left| \frac{f(t)-f(x)}{t-x} \right| = \frac{|f(t)-f(x)|}{|t-x|} \leq |t-x|
\]

implies that $f'(t) = \lim_{x \rightarrow t} \frac{f(t)-f(x)}{t-x} = 0$ at all $t$. The only functions with derivatives that are zero everywhere are constant functions.
\end{proof}

\section*{Problem 2}
A function $f: (a, b) \rightarrow \R$ satisfies a Holder condition of order $\alpha$ if $\alpha > 0$, and for some constant $H$ and all $u, x \in (a, b)$ se have

\[
|f(u) - f(x)| \leq H|u - x|^\alpha
\]

The function is said to be $\alpha$-Holder, with $\alpha$-Holder constant H.

\subsection*{Part a}
Prove that the $\alpha$-Holder function defined on $(a, b)$ is uniformly continuous and infer that it extends uniquely to a continuous function defined on $[a, b]$. Is the extended function $\alpha$-Holder?

\begin{proof}
Let $\epsilon > 0$ and define $\delta = (\frac{\epsilon}{H})^{1/\alpha}$. Then for all $u, x \in (a, b)$ such that $|u-x| < \delta$, we have
\[
|f(u)-f(x)| \leq H|u-x|^\alpha < \epsilon
\]
since $\alpha > 0$.
\end{proof}

By Problem 54 in Chapter 2, a uniformly continuous function defined on a metric space $S$ has a unique continuous extension on $\bar{S}$. Since $[a, b] = \bar{(a, b)}$, $f: (a, b) \rightarrow \R$ being uniformly continuous implies that $f$ extends uniquely to $g: [a, b] \rightarrow \R$, where $g$ is continuous. In fact, $g$ is uniformly continuous because it is continuous on a compact.

We claim that $g$ is $\alpha$-Holder on $[a, b]$. Let $x, y \in [a, b]$. If $x, y \in (a, b)$, this just follows because $g$ extends $f$.

Without loss of generality, let $x = a$ and let $y \in (a, b)$. Let $\epsilon > 0$ be fixed and arbitrary, and let $\delta>0$ be the corresponding continuity condition. Then

\[
|g(c) - g(a)| \leq |g(c) - g(a+\delta)| + |g(a) - g(a+\delta)|
\]

by the Triangle inequality. For the first term, because $c$ and $a+\delta$ are in the interval $(a, b)$, the Holder condition from $f$ extends to $g$, so

\[
|g(c) - g{f}(a+\delta)| \leq H|c-a-\delta|^\alpha \leq H|c-a|^\alpha
\]

because $\alpha > 0$ and $\delta > 0$. For the second term, continuity of $g$ means $|g(a) - g(a+\delta)| < \epsilon$. Thus

\[
|g(c) - g(a)| \leq H|c-a|^\alpha + \epsilon
\]

and $\epsilon$ can be made arbitrarily small. The case where $y = b$, and the case where $x=a$ and $y=b$ simultaneously, are essentially the same.

\subsection*{Part b}

What does $\alpha$-Holder continuity mean when $\alpha = 1$?

When $\alpha=1$, $\alpha$-Holder continuity simplifies to Lipschitz continuity.

\subsection*{Part c}

Prove that $\alpha$-Holder continuity when $\alpha > 1$ implies that $f$ is constant.

Let $x$ in the domain of $f$ be arbitrary. Dividing both sides by $|u-x|$,

\[
0 \leq \frac{|f(u)-f(x)|}{|u-x|} \leq H|u-x|^{\alpha-1}
\]

Let $u \rightarrow x$. Since $\alpha > 1$ the right side goes to $0$, implying $\frac{|f(u)-f(x)|}{|u-x|} \rightarrow 0$ and that $f'(x) = 0$ for all $x$ in $f$'s domain. The only functions with this property are constant functions.

\section*{Problem 3}

Assume that $f:(a, b) \rightarrow \R$ is differentiable.

\subsection*{Part a}

If $f'(x) > 0$ for all $x$, prove that $f$ is strictly monotone increasing.

\begin{proof}
Let $c, d \in (a, b), c < d$. Then because $f$ is differentiable on its domain, the Mean Value Theorem indicates that there is a point $\theta \in (c, d)$ such that

\[
f(c)-f(d) = f'(\theta)(d-c)
\]

Since $f'$ is always strictly positive and $c < d$, the right side is strictly positive.
\end{proof}

\subsection*{Part b}

If $f'(x) \geq 0$ for all $x$, what can you prove?

We can prove that $f$ is weakly monotone increasing. The proof is the same, except that $f'(\theta)(d-c)$ 
can be zero.

\section*{Problem 4}
Prove that $\sqrt{n+1} - \sqrt{n} \rightarrow 0$ as $n \rightarrow \infty$.

Consider the function $f(x) = \sqrt{x}$, and take a Taylor approximation of degree zero around $x = n$, where $n$ is a positive natural number. Then $P_0(x) = \sqrt{n}$. Use the Taylor approximation to approximate $x = n+1$. The Taylor remainder term is

\[
R(1) = \sqrt{n+1} - \sqrt{n}
\]

$\sqrt{x}$ is smooth when $x > 0$, and $n \geq 1$. Therefore, $f$ is smooth on $(n, n+1)$, and the Taylor approximation theorem states that there exists $\theta \in (n, n+1)$ such that

\[
R(1; n) = \sqrt{n+1} - \sqrt{n} = \frac{f'(\theta)}{1!}(1)^1 = \frac{1}{2}\theta^{-\frac{1}{2}}
\]

As $n \rightarrow \infty$, $\theta > n$ implies $\theta \rightarrow \infty$ implies $R(1; n) \rightarrow 0$ implies $\lim_{n \rightarrow \infty} \sqrt{n+1} - \sqrt{n} = 0$.

\section*{Problem 8}

\subsection*{Part b}

Find a formula for a continuous function defined on $[0, 1]$ that is differentiable on the interval $(0, 1)$, but not at the endpoints.

Consider the function

\[
f(x) = 
\begin{cases}
x\sin(\frac{1}{x}) & x \in (0, 1]\\
0 & \text{else}
\end{cases}
\]

$f$ is the composition of continuous functions on $(0, 1]$, so it is continuous on that interval. At $x=0$, we noting that for all $x \in (0, 1]$, we have

\[
-x \leq x \sin(\frac{1}{x}) \leq x
\]

implying that $\lim_{x \rightarrow 0^+} f(x) = 0 = f(0)$ by the Squeeze theorem. This implies that $f(x)$ is continuous at $x=0$, and thus $[0, 1]$. $\frac{1}{x}$ is differentiable on $\R - {0}$, so $f(x)$ is differentiable on $(0, 1]$.

Taking the definition of derivative to attempt to evaluate $f'(0)$,

\[
f'(0) = \lim_{x \rightarrow 0^+} \frac{f(x) - f(0)}{x - 0} = \lim_{x \rightarrow 0^+} \sin(\frac{1}{x})
\]

which does not exist. Thus $f(x)$ is differentiable on $(0, 1]$.

Consider the function

\[
g(x) = f(x) + f(1-x)
\]

This consists of $f$ and $f$ reflected about the line $x = \frac{1}{2}$ added together. From the above, $g$ is continuous on $[0, 1]$, and differentiable on $(0, 1)$, but not $0$ or $1$.

\subsection*{Part c}

Does the Mean Value Theorem apply to such a function?

Yes, since the Mean Value Theorem only requires the function to be differentiable on the open interval. In this case, the Mean Value Theorem states there is a point $\theta \in (0, 1)$ such that $g'(\theta) = 0$. We can probably prove that a point exists by using the Intermediate Value Theorem on $g'(x)$ since it's continuous on $(0, 1)$, but I'm too lazy at the moment.

\section*{Problem 10}

Concoct a function $f: \R \rightarrow \R$ with a discontinuity of the second kind at $x = 0$ such that $f$ does not have the intermediate value property there. Infer that it is incorrect to assert that functions without jumps are Darboux continuous.

Consider the function
\[
f(x) = 
\begin{cases}
x & x \in \R - \Q \\
1 & \text{else}
\end{cases}
\]

$f$ is continuous at $x=1$ and discontinuous everywhere else. These discontinuities are discontinuities of the second kind, since left and right limits don't exist when $x$ is not $1$. $f(x)$ clearly does not have the intermediate value property, as except for $1$, $f$ assumes no rational values. Since this is a function without jump discontinuities but does not possess the intermediate value property, functions without jumps are not necessarily Darboux continuous.

\section*{Problem 11}

Let $f: (a, b) \rightarrow \R$ be given.

\subsection*{Part a}

If $f''(x)$ exists, prove that

\[
\lim_{h \rightarrow 0} \frac{f(x-h) - 2f(x) + f(x+h)}{h^2} = f''(x)
\]

Denote $F(x) = \lim_{h \rightarrow 0} \frac{f(x-h) - 2f(x) + f(x+h)}{h^2}$. Since $f$ is twice differentiable, we take take a second-order Taylor expansion of $f$ around $x$, getting

\[
f(x+h) = f(x) + hf'(x) + \frac{1}{2} h^2 f''(x) + R(x)
\]

where $R(x)$ is second-order flat at $h = 0$, i.e. $\lim_{h \rightarrow 0} R(x)/h^2 = 0$. Similarly,

\[
f(x-h) = f(x) - hf'(x) + \frac{1}{2} h^2 f''(x) + S(x)
\]

where $S(x)$ is second-order flat at $h = 0$. Substituting,

\[
F(x) =
\lim_{h \rightarrow 0} \frac{h^2 f''(x) + R(x) + S(x)}{h^2}
= f''(x)
\]

since the $f(x)$ and $hf'(x)$ terms cancel, and $R(x)$ and $S(x)$ are second-order flat.

\subsection*{Part b}

Find an example that this limit can exist even when $f''(x)$ fails to exist.

Let $f(x) = x|x|$. Taking the first derivative, when $x>0$, $f(x) = x^2$, so $f'(x) = 2x$. Similarly, when $x<0$, $f'(x) = -2x$. When $x=0$,

\[
f'(0) = \lim_{h \rightarrow 0} \frac{f(0+h) - f(0)}{h} = \lim_{h \rightarrow 0} \frac{h|h|}{h} = \lim_{h \rightarrow 0} |h| = 0
\]

Thus

\[
f'(x) =
\begin{cases}
2x & x \geq 0 \\
-2x & x < 0
\end{cases}
\]

As previously stated, $f''(0)$ does not exist, since

\[
f''(0) = \lim_{h \rightarrow 0} \frac{f'(x+h) - f'(x)}{h} = \lim_{h \rightarrow 0} \frac{f'(h)}{h}
\]

which does not exist, since the limit from the positive direction is $2$ and the limit from the negative direction is $-2$.

Despite this, the partial difference approximation exists at $x = 0$. The partial difference approximation from the right is

\[
\lim_{h \rightarrow 0^+} \frac{f(-h) + f(h)}{h^2} = 
\lim_{h \rightarrow 0^+} \frac{-h|-h| + h|h|}{h^2} =
\lim_{h \rightarrow 0^+} \frac{0}{h^2} = \infty
\]

Similarly,

\[
\lim_{h \rightarrow 0^-} \frac{f(-h) + f(h)}{h^2} = 
\lim_{h \rightarrow 0^-} \frac{h|h| + -h|-h|}{h^2} =
\lim_{h \rightarrow 0^-} \frac{0}{h^2} = \infty
\]

Thus the difference approximation exists at $x=0$, even though $f''(0)$ does not exist.

\section*{Problem 15}

Define $f(x) = x^2$ if $x < 0$ and $f(x) = x + x^2$ if $x \geq 0$. Differentiation gives $f''(x) = 2$. This is bogus. Why?

By the Fundamental Theorem of Calculus, if $G$ is an antiderivative of $g$, then $g$ equals the derivative of $G$ where $g$ is continuous. In this case, the standard power rule only applies when $x \neq 0$, since there is a discontinuity there.

Specifically, we have $f''(0)$ does not exist, since $f'(x) = 2x$ when $x \geq 0$, and $f'(x) = 2x + 1$ when $x < 0$. $f'(x)$ is discontinuous at $x=0$, so its derivative does not exist there.

\section*{Problem 16}

$\log x$ is defined to be $\int_1^x 1/t dt$ for $x > 0$. Using only the mathematics explained in this chapter,

\subsection*{Part a}

Prove that $\log$ is a smooth function.

By the Fundamental Theorem of Calculus, the indefinite integral of a Riemann integrable function is continuous with respect to $x$. Thus, $\log x$ is continuous. Its derivative, again by the Fundamental Theorem of Calculus, is $\frac{d}{dx} \int_1^x 1/t dx = 1/x$ when $x > 0$, which is continuous. $1/x$ itself is smooth, so it has derivatives of all orders, which are continuous. Thus $\log x$ is smooth.

\subsection*{Part b}

Prove that $\log(xy) = \log x + \log y$ for all $x, y > 0$.

For any given $y > 0$, define $f(x) = \log xy - \log x - \log y$. By definition,

\begin{align*}
f(x) &= \int_1^{xy} 1/t dt - \int_1^x 1/t dt - \int_1^y 1/t dt \\
&= \int_x^{xy} 1/t dt - \int_1^y 1/t dt
\end{align*}

When $x = 1$, $f(x) = \int_1^{y} 1/t dt - \int_1^{y} 1/t dt = 0$.

We now evaluate $f'(x)$. Splitting the integrals, for all $x>0$, we can find a constant $0 < c < x$. Then

\[
f(x) = \int_c^{xy} 1/t dt - \int_c^x 1/t dt - \int_1^y 1/t dt
\]

By the Fundamental Theorem of Calculus, $\frac{d}{dx} \int_c^x 1/t dt = 1/x$ since $1/t$ is continuous on $[c, \infty)$. By the Chain Rule, $\frac{d}{dx} \int_c^{xy} 1/t dt = y\frac{1}{xy} = 1/x$. $\int_1^y 1/t dt$ is constant with regards to $x$, and thus has derivative zero. Thus, $f'(x) = 0$ for all $x > 0$. The only functions with derivatives equal to zero everywhere are constant functions, and since $f(1) = 0$, this implies that $f(x) = 0$. Thus $\log xy = \log x + \log y$.

\subsection*{Part c}

Prove that $\log$ is strictly monotone increasing and its range is all of $\R$.

$\frac{d}{dx} \log x = 1/x$, which is strictly positive for all $x > 0$. Thus $\log x$ is strictly monotone increasing.

We know that $\log (1) = 0$. Going to the right, let $a_k = \frac{1}{k}$. Because $\frac{1}{t}$ is decreasing, for all $t \in [k, k+1]$, $\frac{1}{t} \leq a_{k+1}$. Thus because $\sum_{k=2}^\infty a_k = \sum_{k=2}^\infty \frac{1}{k}$ diverges to infinity, by the Integral Test, $\int_1^\infty \frac{1}{t} dt$ diverges to infinity. This means that there is for large $x$, $\log (x) = \int_1^x \frac{1}{t} dt$ can be made arbitrarily large. This implies that when $x \geq 0$, $\log(x)$ takes on all values in $[0, \infty)$.

Going to the left, for $x \in (0, 1]$, $\log(x) = - \int_x^1 \frac{1}{t} dt$. Let $k \in \N$ and consider $\log(\frac{1}{2^k}) = - \int_\frac{1}{2^k}^1 \frac{1}{t} dt$.

To evaluate $\int_\frac{1}{2^k}^1 \frac{1}{t} dt$, consider the partition $P$ such that $x_i = \frac{1}{2^i}$ for $i \in \N$. Thus $x_0 = 1$, $x_1 = \frac{1}{2}$, $x_2 = \frac{1}{4}$, etc. Because $\frac{1}{t}$ is strictly decreasing, the minimum of $\frac{1}{t}$ occurs at the right endpoint of the interval. Thus the lower integral is greater than or equal to

\begin{align*}
&1(\frac{1}{2}) + 2(\frac{1}{4}) + 4(\frac{1}{8}) \dots \\
=&\frac{1}{2} + \frac{1}{2} + \frac{1}{2} \dots \\
=&\frac{k}{2}
\end{align*}

because there are $k$ intervals. Since $\frac{1}{t}$ is Riemann integrable on $(0, 1]$, $\frac{k}{2}$ is a lower bound for the integral. Thus

\[
- \int_\frac{1}{2^k}^1 \frac{1}{t} dt \leq -\frac{k}{2}
\]

which implies that the integral goes to negative infinity as $k$ goes to infinity. Thus

\[
-\int_0^1 \frac{1}{t} dt = -\infty
\]

which implies that as $x$ approaches zero, $\log(x)$ approaches negative infinity. Thus on $(0, 1]$, $\log(x)$ takes on all values in $(-\infty, 0]$. Putting the two statements together implies that the range of $\log(x)$ is all of $\R$.

\section*{Problem 29}

Prove that the interval $[a, b]$ is not a zero set.

\subsection*{Part a}

Explain why the following observation is not a solution to the problem: "Every open interval that contains $[a, b]$ has length $> b-a$."

This 'solution' does not consider the possibility that there is a union of open sets that cover $[a, b]$ such that their sum of their lengths can be made arbitrarily small.

\subsection*{Part b}

Instead, suppose there is a "bad" covering of $[a, b]$ by open intervals $\{I_i\}$ whose total length is $< b-a$, and justify the following steps in the proof by contradiction.

I will define a good covering as a covering of $[a, b]$ by open intervals $\{J\}$ such that the total length of the intervals in $\{J\}$ is greater than or equal to $b-a$.

\subsubsection*{i}

It is enough to deal with finite bad coverings.

Let $\{I\}$ be an infinite bad covering of $[a, b]$. Because $\{I\}$ is an open cover of compact $[a, b]$, it reduces to a finite subcovering $\{I_i\}$. Thus, either $\{I\}$ reduces to a finite bad covering, or it reduces to a good covering. If $\{I\}$ reduces to a good covering $\{J_i\}$, then  $\{J_i\} \subset \{I\}$ and the sum of the intervals in $\{J_i\}$ being $\geq b-a$ implies that the sum of the intervals in $\{I\}$ is $\geq b-a$. Thus $\{I\}$ is an infinite good covering, which contradicts the assumption that $\{I\}$ is a bad covering.

Thus, if $\{I\}$ is an infinite bad covering, it reduces to a finite bad covering. Contrapositively, if there are no finite bad coverings, then there are no infinite bad coverings, and the theorem is proven.

\subsubsection*{ii}

Let $\B = \{I_1, \dots I_n\}$ be a bad covering such that $n$ is minimal among all bad coverings.

There is at least one finite bad covering, by assumption. $n=1$ is a lower bound for the size of bad coverings. Then because $\R$ is complete, there exists a greatest lower bound for the sizes of the bad coverings, denoted $c$.

The must be a finite bad covering $\{C\}$ such that the size of $|\{C\}| = c$. Suppose not. Then all bad coverings have size $> c$, and size the sizes of the bad coverings must be integers, all bad coverings have size $\geq c+1$. This contradicts the assumption that $c$ is a greatest lower bound. This bad covering $\{C\}$ is the bad covering with minimal $n$ among all bad coverings.

\subsubsection*{iii}

Show that no bad covering has $n=1$ so we have $n \geq 2$.

This follows from the observation in Part a.

\subsubsection*{iv}

Show that it is no loss of generality to assume $a \in I_1$ and $I_1 \cap I_2 \neq \emptyset$.

There exists at least one interval such that $a \in I_j$, and we are free to denote that interval $I_1$.

There must exist an interval that intersects $I_1$. Suppose not. Let $d_1$ be the right endpoint of $I_1$, and let $c_2, c_3 \dots c_n$ be the left endpoints of the other intervals in the bad covering, and let $c = \min\{c_1 \dots c_n\}$. Then $\frac{c-d}{2}$ is not covered by the bad covering, contradicting the assumption that $\{I\}$ is a covering. Thus, there exists an interval in $\{I\}$ that intersects $I_1$. Denote it $I_2$. By construction, $I_1 \cap I_2$ is nonempty.

\subsubsection*{v}

Show that $I = I_1 \cup I_2$ is an open interval and $|I| < |I_1| + |I_2|$.

If $I_1 \subset I_2$ or $I_2 \subset I_1$, $I_1 \cup I_2$ is trivially an open interval. Otherwise, $I_1 \cup I_2$ is the open because it is the union of open sets, connected because it is the union of two connected sets with a common point, and bounded because it is the finite union of bounded sets. Therefore $I_1 \cup I_2$ is a open, connected, and bounded subset of $\R$, and by the theorems shown in Chapter 2 Problem 31, open, connected, and bounded subsets of $\R$ are open intervals.

\begin{lemma}
Let $C, D \subset \R$ be (bounded) intervals that intersect, and let $E = C + D$. Then $|E| < |C| + |D|$.

\begin{proof}
If $C$ is a subset of $D$ or vice versa, the proof is trivial. Without loss of generality, let the left endpoint of $C$ be less than the left endpoint of $D$. Denote $c$ as the right endpoint of $C$, and $d$ the left endpoint of $D$. $d < c$, otherwise the two intervals do not intersect. Letting $\epsilon = c - d > 0$, the total length of $E$ is $|C| + |D| - \epsilon$, which is strictly less than $|C| + |D|$.
\end{proof}
\end{lemma}

By using the above Lemma, we see that $|I| < |I_1| + |I_2|$.

\subsubsection*{vi}

Show that $\B' = \{I, I_3, \dots I_n\}$ is a bad covering of $[a, b]$ with fewer intervals, contradicting the minimality of $n$.

Let $x \in [a, b]$. Since $\B$ is a covering of $[a, b]$, there exists $i \in 1, 2 \dots n$ such that $x \in I_i$. If $i \geq 3$, then because $I_i \in \B'$, $x$ is also covered by $\B'$. If $i = 1, 2$, then $x \in I = I_1 \cup I_2$, so $x$ is still covered by $\B'$. $\B'$ is a covering by open intervals, because $I$ is an open interval. $\B'$ is a bad covering. $|I| < |I_1| + |I_2|$ implies that $|I| + \sum_{j=3}^n I_j < \sum_{i=1}^n I_i < b-a$, implying that the total length of $\B'$ is less than the total length of $\B$. Thus $\B'$ is a bad covering with fewer intervals than $\B$, contradicting the assumption that $\B$ is the minimal bad covering. Thus, there are no bad coverings of $[a, b]$, coverings of $[a, b]$ can not have arbitrarily small length, and $[a, b]$ is not a zero set.

\section*{Problem 34}

Assume that $\psi: [a, b] \rightarrow \R$ is continuously differentiable. A critical point of $\psi$ is an $x$ such that $\psi'(x) = 0$. A critical value is a number $y$ such that for at least one critical point $x$ we have $y = \psi(x)$.

\subsection*{Part a}

Prove that the set of critical values is a zero set. (This is the Morse-Sard Theorem in dimension one.)

I will first introduce some notation. Let $f: [a, b] \rightarrow \R$ be continuous. I will define a \textbf{zero of type 1} of $f$ to be a zero of $f$ such that $f$ is uniformly zero in an open neighborhood of the root. In other words, if $f(x) = 0$, then there exists $\epsilon > 0$ such that for all $y$ such that $|x-y| < \epsilon$, $f(y) = 0$. I will denote a \textbf{zero point of type 2} of $f$ as all other zeros of $f$. It's clear that the disjoint union of zeros of types 1 and 2 make up all zeros of $f$.

Let $\psi$ be continuously differentiable. We will characterize the critical values of $\psi$ based on the zeros of type 1 and 2 of $\psi'$. If $\psi(x) = y$ is a critical value and $\psi'(x)$ is a zero of type 1, we say that $y$ is a \textbf{critical value of type 1} of $f$. Similarly, if $\psi(x) = y$ is a critical value and $\psi'(x)$ is a zero of type 2, we say that $y$ is a \textbf{critical value of type 2} of $f$. Since the zeros of type 1 and 2 partition the set of zeros of $\psi'$, the critical values of type 1 and 2 partition the set of critical values of $\psi$.

I have no idea if this characterization is standard, but that's what I've come up with.

The immediate characterization for zeros of type 2 is stated below.

\begin{lemma}
\label{ZeroType2ImpliesNonzeroPoint}
Let $f$ be continuous, and let $x$ be a zero of type 2 of $f$. Then for all $\epsilon > 0$, there exists $y \in (x-\epsilon, x + \epsilon)$ such that $f(y) \neq 0$.
\begin{proof}
If this is not true, then $x$ is a zero of type 1.
\end{proof} 
\end{lemma}

To begin with zero points of type 2, we next state a lemma on non-zero points of continuous functions implying an interval with no zeros. This can be thought of as non-zero points of continuous functions creating 'exclusion zones' with a delta-radius that contain no zeros.

\begin{lemma}
\label{ExclusionZoneDelta}
Let $f: [a, b] \rightarrow \R$ be continuous, and let $x \in [a, b]$ be a point such that $f(x) \neq 0$. Then there exists a $\delta > 0$ such that $f$ has no zeros in $(x-\delta, x+\delta)$.
\begin{proof}
Because $f$ is uniformly continuous, there exists a $\delta > 0$ such that for all $y$ such that $|x-y| < \delta$, $|f(x) - f(y)| < |f(x)|$. This implies that $y$ is not a zero of $f$.
\end{proof}
\end{lemma}

\begin{lemma}
\label{LemmaType2NotCluster}
Let $f: [a, b] \rightarrow \R$ be continuous, and let $x \in [a, b]$ such that $f(x) \neq 0$. Then $f$ is not a clustering point of zeros. In other words, we can reasonably speak of the nearest zero of $f$ greater than $x$, and the nearest zero of $f$ less than $x$.
\begin{proof}
Suppose not. Then there exists a sequence $(x_n) \rightarrow x$ of zeros of $f$. $\lim_{n \rightarrow \infty} f(x_n) = 0$, but $f(x) \neq 0$, violating the continuity of $f$.
\end{proof}
\end{lemma}

We now introduce some useful terminology (that I have no idea whether is standard, but I am going to use it). Let $f: [a, b] \rightarrow \R$ be continuous, and let $x$ be a point such that $f(x) \neq 0$. The \textbf{covering interval of x} is the open interval between the nearest zero of $f$ to the left of $x$, and the nearest zero of $f$ to the right of $x$. This interval covers $x$. By Lemma \ref{LemmaType2NotCluster}, this is a well-defined construction.

I will denote the interval as $C$. If there are no zeros of $f$ to the left of $x$, then the left endpoint of $C$ is $a$, and if there are no zeros to the right of $x$, then the right endpoint of $C$ is $b$.

\begin{lemma}
Let $f: [a, b] \rightarrow \R$ be continuous. Then the covering intervals of $f$ are disjoint.
\begin{proof}
Any two covering intervals are separated by at least one zero of $f$.
\end{proof}
\end{lemma}

We now show that the number of covering intervals is closely related to the number of zeros of type 2.

\begin{lemma}
Let $x$ be a zero of type 2. Then there is a covering interval such that $x$ is the endpoint. 
\begin{proof}
Suppose not. Then $x$ is not the nearest zero of type 2 to any nonzero point of $f$. This means that $x$ is a clustering point of zeros of type 2, which contradicts Lemma \ref{LemmaType2NotCluster}.
\end{proof}
\end{lemma}

\begin{corollary}
\label{CorollaryZeroType2Cardinality}
The set of zeros of type 2 is of equal or lesser cardinality to the set of covering intervals.
\begin{proof}
Each zero of type 2 belongs to at least one covering interval.
\end{proof}
\end{corollary}

We now state a brief lemma on exclusion zones and covering intervals.

\begin{lemma}
Let $f: [a, b] \rightarrow \R$ be continuous. Let $x$ be such that $f(x) \neq 0$, and let $\delta$ be the corresponding delta from uniform continuity, as in Lemma \ref{ExclusionZoneDelta}. Let $C$ be the covering interval of $x$. Then the length of the covering interval $C$ is greater than $2\delta$.
\begin{proof}
By Lemma \ref{ExclusionZoneDelta}, there are no zeros of $f$ within the $\delta$-ball centered at $x$. This implies that the nearest zeros of $f$ are further away than $\delta$ in each direction of $x$, implying that the covering interval of $x$ is longer than $2\delta$.
\end{proof}
\end{lemma}

We next show a monotonicity result for covering interval sizes.

\begin{lemma}
\label{CoveringIntervalDelta}
Let $f: [a, b] \rightarrow \R$ be continuous. Let $x$ be such that $f(x) \neq 0$, and let $\delta_x$ be the corresponding delta from uniform continuity. Let $y$ be such that $|f(y)| \geq |f(x)|$. Then the length of the covering interval of $y$ is greater than $2\delta_x$. Covering intervals are weakly increasing with $|f(x)|$.
\begin{proof}
Within the $\delta_x$-ball of $y$, $f$ changes by at most $|f(x)|$, which is insufficient for $f$ to get back to zero.
\end{proof}
\end{lemma}

We now want to prove the main theorem for zeros of type 2. Intuitively, we will show that there are at most countable number of covering intervals, implying that there are most a countable number of zeros of type 2.

\begin{theorem}
Let $f$ be continuous on $[a, b]$. Then $f$ has at most countable zeros of type 2.
\begin{proof}
Consider the set of points where $|f(x)| \geq 1$, and let $A_1$ be the set of covering intervals of those points, with duplicates removed. By Lemma \ref{CoveringIntervalDelta}, there exists $\delta_1 > 0$ such that the covering intervals in $A_1$ have length $\geq 2\delta_1$. Since the covering intervals are disjoint, there are at most $\frac{b-a}{2\delta_1}$ intervals in $A_1$.

We can then consider the set of points where $\frac{1}{2} \leq |f(x)| < 1$, and define $A_2$ to be the set of non-duplicate covering intervals that cover those points. Then there exists $\delta_2 > 0$ such that the covering intervals in $A_2$ have length $\geq 2 \delta_2$, implying that there are at most $\frac{b-a}{2 \delta_2}$ intervals in $A_2$.

Thus, for all $k \in \N$, we can consider the set of points where $\frac{1}{k} \leq |f(x)| < \frac{1}{k-1}$, and define $A_k$ to be the set of non-duplicate covering intervals that cover those points. Continuing in this way for all $k \in \N$, we see that for all $k$, there are countably many covering intervals in $A_k$. Let $A$ be the union of all these sets, and thus the set of all covering intervals. $A$ is countable. By Corollary \ref{CorollaryZeroType2Cardinality}, the set of zeros of type 2 is at most countable.
\end{proof}
\end{theorem}

\begin{corollary}
Let $f$ be continuously differentiable on $[a, b]$. Then $f$ has at most countable critical values of type 2.
\begin{proof}
$f'$ is continuous, implying that $f$ has at most countable zeros of type 2. Each zero of type 2 of $f'$ maps to at most one critical value of type 2 of $f$.
\end{proof}
\end{corollary}

We now turn to critical points of type 1. We first state a useful characterization of zeros of type 1.

\begin{lemma}
\label{LemmaZerosType1}
Let $f: [a, b] \rightarrow \R$ be continuous, and let $Z$ be the set of zeros of type 1. Then $Z$ is the  disjoint union of countable open intervals, with perhaps one or two half-open intervals at the endpoints $a$ and $b$.
\begin{proof}
The definition of zeros of type 1 implies that $Z$ is an open set in $[a, b]$. By the Inheritance Principle, there exists a set $W \subset \R$ that is open in $\R$ such that $W \cap \R = Z$. By Problem 31 in Chapter 2, an open set in $\R$ can be expressed as the disjoint union of countably many open intervals. Taking $W \cap Z$, open intervals that do not contain the endpoints $a$ and $b$ are still open in $Z$, while the half-intervals that have their closed end at $a$ and $b$ become open in $[a, b]$.
\end{proof}
\end{lemma}

We next state a lemma on critical points of type 1.

\begin{lemma}
\label{LemmaCriticalValuesType1}
Let $x$ be a critical point of type 1 for $\psi'(x)$. Then on the neighborhood where $\psi'(x) = 0$, there is only one critical value. Specifically, if $\psi'(x) = 0$ on an interval $(c, d) \subset [a, b]$, then $\psi(c)$ is the only critical value on that interval.
\begin{proof}
By the Fundamental Theorem of Calculus, for $x \in [c, d]$, $\psi(x) = \psi(c) + \int_c^x \psi'(x) dx = \psi(c)$ since $\psi'(x) = 0$ on the interval. 
\end{proof}
\end{lemma}

We now want to prove the main theorem for critical values corresponding to critical points of type 1.

\begin{theorem}
If $f: [a, b] \rightarrow \R$ is continuously differentiable, then $f$ has countably many critical values of type 1.
\begin{proof}
$f'$ is continuous by assumption. By Lemma \ref{LemmaZerosType1}, $Z$, the set of zeros of type 1 of $f'$, consists of countable disjoint open intervals, with perhaps one or two half-open intervals at $a$ and $b$. By Lemma \ref{LemmaCriticalValuesType1}, each (half)-open interval in $Z$ corresponds to one critical value in $f$. Thus $f$ has at most countably many critical values of type 1.
\end{proof}
\end{theorem}

\begin{theorem}
The critical values of $f$ form a zero set.
\begin{proof}
The union of countable sets is a countable set, which is a zero set.
\end{proof}
\end{theorem}

\end{document}