\documentclass{article}
\usepackage{amsmath}
\usepackage{amsfonts}
\usepackage{amssymb}

\newenvironment{proof}{\paragraph{Proof:}}{\hfill$\square$}
\newtheorem{theorem}{Theorem}
\newtheorem{lemma}[theorem]{Lemma}
\newtheorem{corollary}[theorem]{Corollary}

\newcommand{\diam}{\text{diam}}
\newcommand{\N}{\mathbb{N}}
\newcommand{\Z}{\mathbb{Z}}
\newcommand{\R}{\mathbb{R}}

\newcommand{\dist}{\text{dist}}

\newcommand{\overbar}[1]{\mkern 1.5mu\overline{\mkern-1.5mu#1\mkern-1.5mu}\mkern 1.5mu}

\author{Arthur Chen}
\title{Pugh Chapter 2}
\date{\today}

\begin{document}

\section*{Chapter 2 A Taste of Topology}

\subsection*{Problem 6}

Determine whether $d_x(p, q) = \sin |p - q|$ on $[0, \frac{\pi}{2})$ is a metric.

\begin{proof}

It is a metric. Positive definiteness follows from the fact that the absolute value function is a metric over $\mathbb{R}$, and sine being one-to-one over the range of possible functions. Symmetry follows for the same reason. The triangle inequality follows from sine being increasing and concave over $[0, \frac{\pi}{2})$.

Specifically, let $p, r \in [0, \frac{\pi}{2})$, and without loss of generality, let $p \leq r$. If $q = p$ or $q = r$, the triangle inequality is trivial. 

Let $q \notin [p, r]$. If $q > r$, then $q-p > r-p$ implies

\[
d_s(p, q) + d_s(q, r) \geq d_s(p, r) + 0 = \geq d_s(p, r)
\]

A similar result holds if $q < p$. If $q \in (p, r)$, imagine p, q, and r arranged on a line, with p at the origin. As x increases from q to r, the increase in sine is less than the corresponding increase from 0 to r-q, because sine is concave. Thus $\sin(r-p) < \sin(r-q) + \sin(q-p)$.

\end{proof}

\subsection*{Problem 12}

Let $(p_n)$ be a sequence and $f: \mathbb{N} \rightarrow \mathbb{N}$ be a bijection. The sequence $(q_k)_{k\in\mathbb{N}}$ is a rearrangement of $(p_n)$ with $q_k = p_{f(k)}$.

\subsubsection*{Part a}

Are the limits of a sequence unaffected by rearrangement?

The limits of a sequence are unaffected by rearrangement.

Suppose that $(p_n) \rightarrow p$. Then for arbitrary $\epsilon > 0$, there exists an $n \in \mathbb{N}$ such that $i > n$ implies $d(p_i, p) < \epsilon$. This implies that there are at most finite elements of the sequence $(p_n)$ such that $d(p_n, p) \geq \epsilon$. In the rearrangement $(q_n)$, let $M$ be the smallest integer such that $f^{-1}(k) \leq n$. $M$ exists because $\{1, 2, 3\dots n\}$ is finite. Then for all $i > M$, $d(q_i, p) < \epsilon$, and thus $(q_i) \rightarrow p$.

On the other hand, suppose that $(p_n)$ has no limit. Then for arbitrary $p$, there exists an $\epsilon > 0$ such that for all $n \in \mathbb{N}$, there exists $i > n$ such that $d(p_i, p) > \epsilon$. Letting $A(p)$ be the set of these points. $A$ is infinite, because $A$ being finite implies that $A$ has a maximum element.

By contradiction, assume that that $(q_n)$ has a limit of $q$. Then for all $\epsilon_q > 0$, there exists $n(\epsilon_q) \in \mathbb{N}$ such that $i > n(\epsilon_q)$ implies $d(q_i, q) < \epsilon_q$.

Let $\epsilon_q = \epsilon$. Then the set of points $A_q$ such that $d(q_i, q) > \epsilon$ is a subset of $\{1, 2 \dots n(\epsilon)\}$, and is thus finite. But because $(q_n)$ is a rearrangement of $(p_n)$, $A_q = A$. Thus $A_q$ is both finite and infinite, which is a contradiction.

\subsubsection*{Part b}

What if $f$ is an injection?

At first glance, this question seems nonsensical, as for $n\in\mathbb{N}$ which are not in the preimage of $f(\mathbb{N})$, $q_n$ is undefined. If we define such points to have $q_n = 0$, this question becomes intelligible.

Note that if $A$ is an arbitrary set in $\mathbb{N}$, then $f(A)$ has the same cardinality as $A$, because the restriction of $f$ on $A$ is a bijection between the two.

The first result in Part a does not hold. Let $(p_n) = 1$, and let $f(x) = 2x$. The second result still holds, since the cardinality of the sets remains unchanged.

\subsubsection*{Part c}

What if $f$ is a surjection?

The first result in Part a holds. The cardinality of the image of a function must be less than or equal to the cardinality of the domain of the function. Letting $A = \{1, 2 \dots n\}$ and $B = \{n+1, n+2, n+3, \dots\}$, $f(A)$ is finite. Since $f$ is surjective, it covers $\mathbb{N}$, which is infinite. $A + B = \mathbb{N}$ which is the domain of $f$. If $f(B)$ is finite, then $f(A) + f(B) = f(A+B) = f(\mathbb{N})$ is finite, which contradicts the assumption that $f$ is surjective. Therefore $f(B)$ is infinite, and the argument in Part a holds.

The second result in Part b does not hold. Let $(p_n) = \{0, 0, 0, 1, 0, 1, 0, 1 \dots\}$ and let $f(x) = 1$ if $x$ is odd or equals 2, and $\frac{1}{2}x$ if $x$ is even and greater than 3. $f$ is clearly surjective, and $(q_n) = \{0, 1, 1, 1, 1\dots\}$ clearly has a limit of 1.

\subsection*{Problem 13}

Show that if $f: M\rightarrow N$ is a function such that $(p_n)$ converging in $M$ implies that $(f(p_n))$ converges in $N$, then $f$ is continuous.

Note that this is almost the definition of convergence, except for the requirement that $(f(p_n)) \rightarrow f(p)$. Thus showing that $f$ is continuous is equivalent to showing the above requirement.

\begin{proof}

Let $p\in M$ be arbitrary. Let $(a_n)$ be the uniform sequence $a_n = p$ for all $n\in\mathbb{N}$. From the properties of $f$, $(f(a_n))$ converges in $N$, and it converges to $f(p)$. If $(a_n)$ is the only sequence in $M$ that converges to p for all p in $M$ (such as when the discrete metric is used), then $f$ satisfies the sequential convergence condition and is thus continuous.

Let $(b_n)$ be an arbitrary sequence in $M$ such that $b_n \rightarrow p$. Construct the sequence $(c_n)$ such that $c_n = a_{\lceil n/2 \rceil}$ if n is odd and $c_n = b_{n/2}$ if n is even. $(c_n)$ clearly converges to $p$.

By the convergence preservation condition of $f$, $(f(c_n))$ converges in $N$. The subsequence of $(f(c_n))$ consisting of the odd numbers converges to $f(p)$. All subsequences of a convergent sequence converge to the same limit as the main sequence, so $f(c_n) \rightarrow f(p)$, and since $(f(b_n))$ is a subsequence of $(f(c_n))$, $f(b_n) \rightarrow f(p)$. Thus $f$ preserves sequential convergence, and is thus continuous.

\end{proof}

\subsection*{Problem 14}

Let $f: M\rightarrow N$ be a bijection from one metric space to another that preserves distance, i.e. for all $p, q \in M$

\[
d_N(fp, fq) = d_M(p, q)
\]

Then $f$ is called an isometry from $M$ to $N$, and $M$ and $N$ are said to be isometric, $M \equiv N$.

\subsubsection*{Part a}

Prove that every isometry is continuous.

\begin{proof}

Let $(p_n) \in M$ be an arbitrary sequence such that $p_n \rightarrow p \in M$. Then for arbitrary $\epsilon > 0$, there exists an $A \in \mathbb{N}$ such that $a > A$ implies that $d_M(p_a, p) < \epsilon$. Moving to the sequence $(f(p_n))$, by the distance preservation property we have that for the same $\epsilon > 0$, for the same $a > A$, we have $d_N(f(p_a), f(p)) < \epsilon$. Thus $f(p_n) \rightarrow f(p) \in N$, so $f$ preserves sequential convergence and is thus continuous.

\end{proof}

\subsubsection*{Part b}

Prove that every isometry is a homeomorphism.

\begin{proof}

Since $f$ is a bijection, its inverse $f^{-1}$ exists. Since we proved that $f$ is continuous, if we prove that $f^{-1}$ is continuous, then $f$ is a homeomorphism between $M$ and $N$, and thus $M$ and $N$ are homeomorphic.

Let $(p_n) \in N$ be arbitrary such that $p_n \rightarrow p \in N$. Then because $f$ is a bijection, $f^{-1}$, exists, and so $(f^{-1}(p_n))$ and $f^{-1}(p)$ are well defined.

Suppose $f^{-1}$ is not continuous. Then $(f^{-1}(p_n))$ does not converge to $f^{-1}(p)$, and since $f$ is continuous and the inverse of $f^{-1}$, this implies that $(f(f^{-1}(p_n))) = (p_n)$ does not converge to $f(f^{-1}(p)) = p$. But this contradicts the assumption that $p_n \rightarrow p$. Thus $f^{-1}$ is continuous, and $M$ and $N$ are homeomorphic.

\end{proof}

\subsubsection*{Part c}

Prove that $[0, 1]$ is not isometric to $[0, 2]$.

\begin{proof}

Consider the set of pairs of points in $[0, 2]$ that are distance 1 from each other. Because of the total ordering that $[0, 2]$ inherits from $\mathbb{R}$, these pairs can be uniquely identified by their left endpoints. Letting $A$ be the set of left endpoints of such pairs of points, $A$ takes the form $A = \{a: [a, a+1] \subset [0, 2]\} = [0, 1]$.

Let $f$ be an isometry from $[0, 1]$ to $[0, 2]$. By the distance preservation condition of $f$, for all $a \in A$, the preimage of $a$ under $f$ is a point $b \in [0, 1]$ such that the interval $[b, b+1] \subset [0, 1]$. However, the only interval of such form is $[0, 1]$. $f(0)$ can not correspond to $[0, 1]$, as this violates $f$ being a function. Thus an isometry from $[0, 1]$ to $[0, 2]$ does not exist.

\end{proof}

\subsection*{Problem 15}

Prove that isometry is an equivalence relation.

\begin{proof}
Let $M$ be isometric to $N$, and $f: M\rightarrow N$ be the isometry. For arbitrary $p, q \in N$, consider $f^{-1}(p), f^{-1}(p) \in M$. By the distance preserving condition of $f$, $d_M(f^{-1}(p), f^{-1}(q)) = d_N(f(f^{-1}(p)), f(f^{-1}(q))) = d_N(p, q)$. Thus $f^{-1}$ is a bijection that preserves distances from $N$ to $M$, and thus $N$ is isometric to $M$.

Let $M$ be a metric space and $f$ be the identity function. $f$ is clearly a bijection from $M$ to $M$ and distance-preserving, so $f$ is an isometry. Therefore $M$ is isometric to itself.

Let $f:M \rightarrow N$ and $g:N \rightarrow P$ be isometries. Consider their composition $H:M \rightarrow P = (g \circ f)(m)$. $H$ is the composition of bijective functions, and thus is bijective itself, and it's clear that $H$ preserves distances between $M$ and $P$. Thus $M$ is isometric to $P$.
\end{proof}

\subsection*{Problem 16}

Is the perimeter of a square isometric to the circle?  Homeomorphic?

Assuming that the square and the circle are embedded in $\mathbb{R}^2$ with the usual metric, the two are not isometric. If the length diagonal of the square does not equal the diameter of the circle, the proof is immediate. If the length of the diagonal of the square equals the diameter of the circle, call their common distance $d$. There are only two pairs of points on the square with distance $d$ from each other, while there are an infinite number of pairs on the circle distance $d$ from each other. Therefore $f$ can not map between them while remaining a function.

The two are homeomorpic, as one can be stretched into the other.

\subsection*{Problem 18}

Is $\mathbb{R}$ homeomorphic to $\mathbb{Q}$?

No, as the two have different cardinalities, there can not be a bijection between them.

\subsection*{Problem 19}

Is $\mathbb{Q}$ homeomorphic to $\mathbb{N}$?

No. Let $p \in \mathbb{N}$. All convergent sequences $p_n \rightarrow p$ in $\mathbb{N}$ eventually end in repeating $p$. On the other hand, for $q \in \mathbb{Q}$, there exist sequences $(q_n), (r_n) \in \mathbb{Q}$ such that for all $n$, $q_n \neq r_n \neq q$.

Suppose that $f$ is a homeomorphism from $\mathbb{Q}$ to $\mathbb{N}$ such that $f(q) = p \in \mathbb{N}$.  Since $f$ is bicontinuous, it preserves sequential convergence. Thus $(f(q_n))$ and $(f(r_n))$ both converge to $p$. Because of the nature of continuous sequences in $\mathbb{N}$, there exists an $N \in \mathbb{N}$ such that $n > N$ implies that $f(q_n) = f(r_n) = p$. But this implies that $f$ is not injective, and thus not a bijection, which contradicts the assumption that $f$ is a homeomorphism. Thus there exists no homeomorphism between $\mathbb{Q}$ and $\mathbb{N}$.

\subsection*{Problem 20}

What function is a homeomorphism from $(-1, 1)$ to $\mathbb{R}$? Is every open interval homeomorphic to $(0, 1)$?

The function $\tan(\frac{\pi}{2}x)$ is a homeomorphism from $(-1, 1)$ to $\mathbb{R}$. It is bijective, continuous, and its inverse $\frac{2}{\pi}\tan^{-1}(x)$ is continuous.

Every open interval $(a, b)$ is homeomorphic to $(0, 1)$. The function $\frac{x-a}{b-a}$ is bijective and bicontinuous.

\subsection*{Problem 22}

If every closed and bounded subset of a metric space $M$ is compact, does it follow that $M$ is complete?

No. Let $M = {x\in \mathbb{R}: x > 0}$ with the usual metric. The closed and bounded subsets of $M$ are compact, by the same reasoning as in $\mathbb{R}$. However, the Cauchy sequence $(p_n) = \frac{1}{n}$ has no limit in $M$.

\subsection*{Problem 24}

For which intervals $[a, b]$ in $\mathbb{R}$ is the intersection $[a, b]\cup \mathbb{Q}$ a clopen subset of the metric space $\mathbb{Q}$?

The intervals where $a \leq b$ and $a, b$ are irrational numbers. Since $\mathbb{Q}$ is a metric subspace of $\mathbb{R}$, it inherits its open and closed sets from $\mathbb{R}$. Since all intervals of the form $[a, b]$ are closed in $\mathbb{R}$, their intersection with $\mathbb{Q}$ is closed.

These intersections are also open. In $\mathbb{R}$, the interval $(a, b)$ is open, and because $a, b$ are irrational, $a, b \notin \mathbb{Q}$, so $[a, b]\cup\mathbb{Q} = (a, b)\cup\mathbb{Q}$. Thus $[a, b]\cup\mathbb{Q}$ is open.

The above does not hold if $a$ or $b$ is rational. Without any loss of generality, suppose $a$ is rational. Then $a \in [a, b]\cup\mathbb{Q}$, and there is no open ball that contains $a$ and is a subset of  $[a, b]\cup\mathbb{Q}$. Thus $[a, b]\cup\mathbb{Q}$ is not open.

\subsection*{Problem 26}
Prove that a set $U \subset M$ is open if and only if none of its points are the limits of its complement.

For the forward, suppose $x \in U$ is a limit point of $U^C$. Then $x \in \bar{U^C}$, and since for all sets $A \subset \bar{A}$, $x \in U^C$, which contradicts the assumption that $x \in U$. Thus $U$ contains none of the limits of its complement.

For the reverse, suppose that $U$ is not open. Then there exists a $x \in U$  such that for all $r > 0$, the open ball $B_U(x, r)$ is not fully contained in $U$. Thus, for $r = \frac{1}{n}$, we can choose a sequence $x_n \in B_U(x, r)\cup U^C$. $x_n \rightarrow x$ and $x_n \in U^C$ for all $n$, so $x$ is a limit point of $U^C$. This contradicts the assumption that none of $U$'s points are the limit points of its complement, so $U$ is open.

\subsection*{Problem 27}
If $S, T \subset M$, a metric space, and $S \subset T$, prove that
\subsubsection*{Part a}
$\bar{S} \subset \bar{T}$
\begin{proof}
Let $x \in \bar{S}$. Then there exists a sequence $(x_n) \in S$ such that $x_n \rightarrow x$. Since $S \subset T$, $(x_n) \ in T$, so $x_n \rightarrow x$ implies that $x \in \bar{T}$.
\end{proof}

\subsubsection*{Part b}
$\text{int}(S) \subset \text{int}(T)$
\begin{proof}
$\text{int}(S) \subset S \subset T$. Because $\text{int}(S)$ is open and a subset of $T$, it must be contained in the largest open subset of $T$, which is $\text{int}(T)$.
\end{proof}

\subsection*{Problem 28}

A map $f: M \rightarrow N$ is open if $U$ being an open subset of $M$ implies that $f(U)$ is an open subset of $N$.

\subsubsection*{Part a}
If $f$ is open, is it continuous?

No. Let $f$ be the floor function from $\mathbb{R}$ to $\mathbb{Z}$. Every set in $Z$ is open, since singleton sets are open in $Z$ and the union of open sets is open, whether finite or infinite. However, the floor function is not continuous, as the example of $p_n = 1 - \frac{1}{n}$ shows.

\subsubsection*{Part b}
If $f$ is a homeomorphism, is it open?

Yes. Let $f: M \rightarrow N$. Let $A \subset M$ be an arbitrary open set, and $f(A) \subset N$ be its image. Because $f$ is a bijection, it has a continuous inverse function $f^{-1}$. Since $f(A)$ is the preimage of $A$, $f^{-1}(f(A)) = A$ being open implies $f(A)$ is open. Thus open sets are mapped to open sets, and $f$ is open.

\subsubsection*{Part c}
If $f$ is an open, continuous bijection, is it a homeomorphism?

Yes. By similar reasoning to Part b, let $A \subset M$ be an arbitrary open set. Because $f$ is a bijection, $f^{-1}$ exists, and $A$ is the image of some set $f(A) \subset N$. Because $f$ is open, $f(A)$ is open. Thus the preimage of open sets of $f^{-1}$ is an open set, and $f^{-1}$ is continuous. Since $f$ is a bicontinuous bijection, it is a homeomorphism.

\subsubsection*{Part d}
If $f: \mathbb{R} \rightarrow \mathbb{R}$ is a continuous surjection, must it be open?

No. Let $f$ be defined as $f(x) = x \text{ if } x\in\mathbb{Z}$. For $x \in \mathbb{R} - \mathbb{Z}$, define $f$ as

\[
f(x) = 
\begin{cases}
x & x \in \mathbb{Z} \\
\lfloor x \rfloor + 2\text{Remainder}(x) & \text{Remainder}(x) \in (0, \frac{1}{2}) \\
\lceil x \rceil & \text{Remainder}(x) \in [\frac{1}{2}, 1)
\end{cases}
\]

where $\text{Remainder}(x)$ is the non-integer part of $x$. $f(x)$ is essentially a piecewise staircase with a slope of 2 on $[0, \frac{1}{2})$ pieces and a slope of 0 on $(\frac{1}{2}, 1)$ pieces. The function is continuous and surjective, but not open. For example, the open set $(\frac{1}{2}, \frac{3}{4})$ maps to the singleton ${1}$, which is not open.

\subsubsection*{Part e}
If $f: \mathbb{R} \rightarrow \mathbb{R}$ is a continuous, open surjection, must it be a homeomorphism?

Yes. Note that if $f$ is injective, then $f$ is a bijection and Part c implies that it is a homeomorphism.

Suppose that $f$ is not injective. Then there exists $a, b \in \mathbb{R}, a \neq b$ such that $f(a) = f(b)$. Without loss of generality, let $a < b$. Then by continuity, $f$ achieves a maximum value on $[a, b]$.

If $f(a) = M = m$, then $f$ is constant on $[a, b]$. This contradicts the assumption that $f$ is open. For example, the image of the open ball $(a + \frac{b-a}{4}, b - \frac{b-a}{4})$ is a singleton, and thus not open.

If $f(a) < M$, then there exists $c \in (a, b)$ such that $f(c) = M$. Thus there exists $\delta = \min(\frac{c-a}{2}, \frac{b-a}{2})>0$. Consider the open ball $(c-\delta, c+\delta)$ and its image under $f$. Because $f$ is bounded, the image is bounded, and because $f$ is continuous, the image obtains all intermediate values. Thus $f((c-\delta, c+\delta))$ is an interval. Let $x, y$ be the left and right endpoints of the interval, respectively. 

Since $f(c)$ is the maximum of $f$, the right endpoint of the interval is closed. Thus $f((c-\delta, c+\delta))$ has the form $(x, y]$ or $[x, y]$, neither of which is open. Thus, the image of an open set under $f$ is not open, which contradicts the assumption that $f$ is open. A similar argument holds for the minimum. This, $f$ is injective, bijective, and a homeomorphism.

\subsubsection*{Part f}

What happens in Part e if $\mathbb{R}$ is replaced by the unit circle $S^1$?

The result does not hold. Parameterize $S^1$ by the angle $\theta = \arctan(x)$, and let $f: S^1 \rightarrow S^1$ map a point $x$ to the corresponding point at $2\theta$. $f$ is obviously surjective, and it is continuous because it preserves sequential limits. It is open because open sets in $S^1$ consist of open line segments or their unions, and the image of line segments is either another union of line segments, or the entire metric space $S^1$. However, $f$ is not injective. Letting $a$ be the point at $\theta=0$ and $b$ the point at $\theta = \pi$, $a\neq b$, but $f(a) = f(b)$.

\subsection*{Problem 30}

Consider a two-point set $M = \{a, b\}$ whose topology consists of two sets, $M$ and the empty set. Why does this topology not arise from a metric on $M$?

Let $d$ be a metric on $M$. By the properties of metrics, $d(a, b) = d(b, a) = c > 0$, and $d(a, a) = d(b, b) = 0$. The singleton sets ${a}, {b}$ are open, since the open ball $B_d(a, c/2) = {a}$ and $B_d(b, c/2) = {b}$ are contained within themselves. However, the singleton sets are not contained in the topology.  

\subsection*{Problem 31}
Prove the following.
\subsubsection*{Part a}
If $U$ is an open subset of $\mathbb{R}$ then it consists of countably many disjoint intervals $U = \sqcup U_i$. (Unbounded intervals $(-\infty, b)$, $(a, \infty)$, and $(-\infty, \infty)$ are allowed).

\begin{lemma}
\label{ConnectedSetsAreIntervals}
The bounded, connected sets on $\mathbb{R}$ are intervals.
\begin{proof}
Let $A$ be bounded and connected. By connectedness, it has the intermediate value property. By boundedness, it has a l.u.b. and g.l.b. Letting $a$ be the g.l.b. and $b$ be the l.u.b. and using the intermediate value property, $(a, b) \subset A$. By the l.u.b., $A \subset [a, b]$. Thus $A$ is an interval. The reverse was shown in the book.
\end{proof}
\end{lemma}

\begin{corollary}
\label{OpenConnectedSetsAreOpenIntervals}
The bounded, connected, open sets on $\mathbb{R}$ are open intervals.
\begin{proof}
The connected sets on $\mathbb{R}$ are intervals, and the (topologically) open intervals are (colloquial) open intervals.
\end{proof}
\end{corollary}

\begin{lemma}
\label{OverlappingIntervalsAreIntervals}
Let $A$ and $B$ be overlapping bounded open intervals. Then $A \cup B$ is an open interval.
\begin{proof}
$A$ and $B$ are connected and share a common point, so $A \cup B$ is connected. Since $A \cup B$ is bounded, $A \cup B$ is an interval. Since $A \cup B$ is open, $A \cup B$ is an open interval.
\end{proof}
\end{lemma}

\begin{corollary}
\label{BoundedIntervalsBecomeDisjoint}
Let ${A_i}$ be a collection of disjoint bounded open intervals of $\mathbb{R}$. Then there exists disjoint open intervals ${B_i}$ such that $\cup A_i = \sqcup B_i$
\begin{proof}
If the $A_i$ are disjoint, the proof is obvious. If not, then there exist $i, j$ such that $A_i$ and $A_j$ overlap. By Lemma \ref{OverlappingIntervalsAreIntervals}, $B_i = A_i \cup A_j$ is an open interval. Substitute $A_i$ and $A_j$ with $B_i$, and repeat while there are still overlapping intervals in $A$.
\end{proof}

(I'm not sure that this proof is rigorous. It seems like I'm implicitly assuming that $A_i$ is countable, but I don't know enough to be sure.)
\end{corollary}

\begin{lemma}
\label{OpenSetsAreUnionsOfOpenIntervals}
If $A$ is an open subset in $\mathbb{R}$, then there exist bounded open intervals $A_i$ such that $\cup A_i = A$.
\begin{proof}
Because $A$ is open, for all $a \in A$, there exists $r_a > 0$ such that $(a-r_a, a+r_a) \subset A$. Do this for all points in $A$.
\end{proof}
\end{lemma}

\begin{lemma}
\label{IntervalsGreaterThanEpsilonAreCountable}
Let $\epsilon > 0$. Let $A_i$ be a collection of disjoint open intervals such that $\text{length}(A_i) > \epsilon$ for all $i$. Then $A_i$ is countable.
\begin{proof}
Let $B_i$ be the intervals of $A_i$ dilated by a factor of $1/\epsilon$. That is, if $A_i = (x, y)$, $B_i = (x/\epsilon, y/\epsilon)$. Because all of the $A_i$s have length greater than $1/\epsilon$, the $B_i$ have length greater than 1. Thus the $B_i$ can be uniquely identified with the natural numbers by associating each $B_i$ with the floor of its left endpoint. Since the natural numbers are countable, $B_i$ is at most countable, and there exists a homeomorphism between the $A_i$ and $B_i$, the $A_i$ are countable.
\end{proof}
\end{lemma}

\begin{theorem}
\label{OpenSetsAreUnionOfDisjointIntervals}
If $A$ is an open subset of $\mathbb{R}$, then it consists of disjoint open intervals $A = \sqcup A_i$ (unbounded open intervals are acceptable).

\begin{proof}
If $A$ is bounded, then by Lemma \ref{OpenSetsAreUnionsOfOpenIntervals}, $A$ can be expressed as the union of open intervals, and by Corollary \ref{BoundedIntervalsBecomeDisjoint}, $A$ can be expressed as the disjoint union of bounded open intervals.

If $A$ is unbounded and $A = \mathbb{R}$, the proof is trivial. Suppose $A$ is unbounded in the positive direction. For simplicity, suppose that $A$ is bounded in the negative direction. Since $A$ does not equal $\mathbb{R}$, $A^C$ is nonempty, and by $A$ being unbounded in the positive direction, $A^C$ is bounded above. Thus $A^C$ has a least upper bound. Denote the l.u.b. of $A^C$ as a. By definition, $(a, \infty) \subset A$. Because $A$ is open, $A^C$ is closed, and since $a$ is a limit point of $A^C$, $a \in A^C$.

Let $C = A-(a, \infty)$. For all $c \in C$, $c < a$. If $c = a$, then $C \subset A$ implies $c \in A$, which contradicts that $a \in A^C$. If $c > a$, then $c \in C$ and $c \in (a, \infty)$, which is a contradiction because by construction $C$ and $(a, \infty)$ are disjoint. Thus $C$ is bounded above. $C$ is bounded below because $C \subset A$ and $A$ is bounded below. Thus $C$ is bounded.

$C$ is open. $C^C = A^C \cup (a, \infty) = A^C \cup [a, \infty)$ because $a \in A^C$. Thus $C^C$ is the finite union of closed sets, and is closed.

Thus $C$ is a bounded open subset of $\mathbb{R}$, and can be expressed as the disjoint union of bounded open intervals $C_i$. Adding back in $(a, \infty)$ gives $A$ as the disjoint union of open intervals $A_i$. The cases where $A$ is bounded above and unbounded below, and case when $A$ is unbounded in both directions, follow similarly.
\end{proof}
\end{theorem}

\begin{lemma}
Let $A_i$ be a collection of disjoint intervals in $\mathbb{R}$. Then there is at most one unbounded interval in the positive direction, and and most one (possibly the same) unbounded interval in the negative direction.
\begin{proof}
Let $A_m$ and $A_n$ be distinct intervals in $A$ that are unbounded above. Because they are intervals, they are non-empty, so there exist $m \in A_m$ and $n \in A_n$. By the trichotomy property on $\mathbb{R}$, either $m = n$, $m > n$, or $n > m$. In any of the cases, one of $m$ or $n$ is in both sets, violating the assumption that $A_i$ is disjoint.
\end{proof}
\end{lemma}

\begin{theorem}
\label{DisjointOpenIntervalsAreCountable}
If $A_i$ is a collection of disjoint open intervals in $\mathbb{R}$, then $A_i$ is countable.
\begin{proof}
Create subcollections $B_i$, where $i = 0, 1, 2 \dots \mathbb{N}$. Let $B_0$ consist of the unbounded intervals of $A_i$, of which there are at most 2, $B_1$ the intervals of $A_i$ with finite length greater than 1, and $B_i$ the intervals of $A_i$ with length greater than $1/ i$ and less than or equal to $1/(i-1)$. By Lemma \ref{IntervalsGreaterThanEpsilonAreCountable}, $B_i$ contains at most countable many intervals. Continuing this process for all $\mathbb{N}$, we see that the $B_i$s contains all the intervals of $A_i$. Because $B_i$ contains a countable number of sets, each with a countable number of elements, $B_i$, and thus $A_i$ is countable.
\end{proof}
\end{theorem}

\begin{corollary}
If $U$ is an open subset of $\mathbb{R}$ then it consists of countably many disjoint intervals $U = \sqcup U_i$. (Unbounded intervals $(-\infty, b)$, $(a, \infty)$, and $(-\infty, \infty)$ are allowed).
\begin{proof}
Follows immediately from Theorems \ref{OpenSetsAreUnionOfDisjointIntervals} and \ref{DisjointOpenIntervalsAreCountable}.
\end{proof}
\end{corollary}

\subsubsection*{Part b}
Prove that the intervals $U_i$ are uniquely determined by $U$. In other words, there is only one way to express $U$ as a disjoint union of open intervals.

I will assume that the $U_i$ have to be open, otherwise the statement is false. For example, $(0, 1)$ can also be written as $(0, .5] \cup (.5, 1)$.

\begin{lemma}
\label{EndpointsOfDisjointUnionsAreOutside}
Let $A_i$ be a disjoint union of open intervals and $A = \sqcup A_i$. Let $A_k$ be a particular interval with left and right endpoints $b$ and $c$, respectively. Then $b, c \in A^C$.
\begin{proof}
Suppose $b \in A$. Since $b \in A_k^C$, $b$ must be in a different open interval. Denote the open interval $A_j$. Since $A_j$ is open, there exists $r > 0$ such that $(b-r, b+r) \subset A_j$. Consider $b-r/2$. $b-r/2 < b$, so $b-r/2 \in A_k$. $b-r/2 \in (b-r, b+r) \subset A_j$, so $b-r/2 \in A_j$. Thus $A_k$ and $A_j$ intersect, which contradicts the assumption that the $A_i$ are disjoint. Thus $b \in A^C$. The exact same argument holds for $c$.
\end{proof}
\end{lemma}

Now to prove the theorem. Let $A$ be an open subset and $B_i$ and $C_j$ be disjoint collections of open intervals such that $A = \sqcup B_i = \sqcup C_j$. If $A$ is unbounded in both directions, the proof is trivial.

Consider an arbitrary point $a \in A$. Because the $B_i$'s are disjoint and the $C_j$'s are disjoint, there exist unique $B_a$ and $C_a$ such that $a \in B_a$ and $a \in C_a$.

Letting $x$ be the right endpoint of $B_a$ and $y$ the right endpoint of $C_a$, $x = y$. Suppose it's not. Without loss of generality, let $x < y$. Then $x \in A$ because $x \in C_a$, but $x \in A^C$ because $x$ is the right endpoint of $B_a \subset A$, and so $x \in B_a$. Repeating this for $x > y$ shows that $x = y$.

A similar argument holds for the left endpoint. Repeat this argument for all points in $A$ to show that for all $a \in A$, the endpoints of $B_a$ equal the endpoints of $C_a$. Thus, $B_a = C_a$ and the two collections $B$ and $C$ are equal.

\subsubsection*{Part c}

If $U, V \subset \mathbb{R}$ are both open, so $U = \sqcup U_i$ and $V = \sqcup V_j$ where $U_i$ and $V_j$ are open intervals, show that $U$ and $V$ are homeomorphic if and only if there are equally many $U_i$ and $V_j$.

\begin{proof}

Let $m$ be the number of element in $U_i$, and $n$ be the number of elements in $V_j$. Note that $m, n$ may be infinity.

For the forward, suppose there are not equally many $U_i$ and $V_j$. Without loss of generality, let $m < n$. Let $f$ be a homeomorphism between $U_i$ and $V_j$. For arbitrary $k \in 1, 2 \dots m$, consider the restriction of $f$ on $U_k$. Because $U_k$ is an interval, it is connected, and so its image is connected. Repeating this for all $U_i$ shows that $V$ is the union of at most $m$ connected sets. However, $V$ consists of $n$ connected intervals, so $V$ can not be the image of $U$. Thus the homeomorphism $f$ does not exist. A similar argument in reverse if $m > n$.

For the reverse, construct a homeomorphism as follows. By Part a above, $U_i$ and $V_j$ are countable. Label the sets. Let the homeomorphism be $f$ such that each $U_k$ is shifted and dilated so that $f(U_k) = V_k$. $f$ is trivially a bijection. Since on each segment, $f$ is a non-singular linear transformation, $f$ and $f^{-1}$ are continuous.
\end{proof}

\subsection*{Problem 33}

\subsubsection*{Part a}
Find a metric space in which the boundary of $M_r p$ is not equal to the sphere of radius $r$ at $p$, $\partial(M_r p) \neq {x \in M: d(x, p) = r}$.

Consider the discrete metric on a space with at least two elements. Note that since all sets under the discrete metric are clopen, the boundry of all sets is the null set. However, the sphere of radius 0 centered at $p$ is $p$ itself, and the sphere of radius 1 centered at $p$ is the entire metric space, neither of which are the null set.

\subsubsection*{Part b}

Need the boundary be contained in the sphere?

Yes. Suppose not. For an arbitrary set $A$ in a metric space $M$, $\partial A \subset \overbar{A}$. Thus the boundary not being contained in the sphere implies that $CB_r p = \{x \in M| d(x, p) \leq r\}$ is a proper subset of $\overbar{M_r p}$, so that the boundary can potentially be outside of $CB_r p$. Because $\overbar{M_r p}$ is the smallest closed set that contains $M_r p$ and $M_r p \subset CB_r p$, $CB_r p$ being a proper subset of $\bar{M_r p}$ implies that $CB_r p$ is not closed. This implies that $CB_r p^C = \{x \in M: d(x, p) > r \}$ is not open. If $CB_r p^C$ is the null set, this is a contradiction.

Otherwise, for all $x \in CB_r p^C$, let $d(x, p) = y > r$. Then the open ball $M_{\frac{y-r}{2}} x \subset CB_r p^C$ by the triangle inequality, and $CB_r p^C$ is open, implying $CB_r p$ is closed. Thus there is a closed set containing $M_r p$ that is strictly smaller than the closure of $M_r p$, a contradiction.

\subsection*{Problem 34}

Assume that $N$ is a metric subspace of $M$ and is also a closed subset of $M$. Show that $L \subset N$ is closed in $N$ if and only if it is closed in $M$. Similarly, if $N$ is a metric subspace of $M$ and also is an open subset of $M$ then $U \subset N$ is open in $N$ if and only if it is open in $M$.

\begin{proof}
For the closed version, we start with the forward. Because $L$ is closed in $N$, by the Inheritance Principle, $L = N \cap P$, where $P$ is a closed set in $M$. Viewing $N$ and $P$ as subsets of $M$, since $N$ is closed by assumption, $L = N \cap P$ is closed in $M$. Conversely, let $L$ be closed in $M$. Then by the Inheritance Principle, $L \cap N$ is a closet set in $N$. Since $L \subset N$, $L \cap N = L$ is closed in $N$. The open version is essentially the same, with the word 'open' replacing the word 'closed.'
\end{proof}

\subsection*{Problem 35}
Prove that $S$ clusters at $p$ if and only if for each $r > 0$ there is a point $q \in M_r p \cap S$ such that $q \neq p$.

This should be a direct consequence of Theorem 52, that $p$ being a cluster point of $S$ is characterized by each neighborhood of $p$ containing at least one point of $S$ other than $p$.

\subsection*{Problem 36}
Construct a set with exactly three cluster points.

In $\mathbb{R}^2$, let $A_i = \{(x, y) \in \mathbb{R}^2| (x, y) = (i, 1/n) \text{ for some } n \in \mathbb{N}\}$. Let $A = A_1 \cup A_2 \cup A_3$. The three cluster points are $\{(0, 0), (1, 0), (2, 0)\}$.

\subsection*{Problem 37}

Construct a function $f: \mathbb{R} \rightarrow \mathbb{R}$ that is continuous only at points of $\mathbb{Z}$.

First we consider the function which is $x$ on the rationals and $-x$ on the irrationals. It is continuous only at $x = 0$. Take miniature copies of this function, centered at the integers, all only on the interval $(n - 1/2, n + 1/2]$. For all integers $z$, $f(z) = 0$. The function is continuous at the integers, and nowhere else.

\subsection*{Problem 38}

Let $X, Y$ be metric spaces with metrics $d_X, d_Y$, and let $M = X \times Y$ be their Cartesian product. Prove that the three natural metrics $d_E$, $d_{max}$, and $d_{sum}$ on $M$ are actually metrics.

Symmetry is obvious. The positive definiteness of $d_{max}$ follows from the positive definiteness of the $d_X$ and $d_Y$. The positive definiteness of $d_{sum}$ is similar, and the positive definiteness of $d_{E}$ follows because the square root function only has a root at 0.

For the triangle inequality on $d_{sum}$,

\begin{align*}
d_{sum}(p, p^*) + d_{sum}(p^*, p') &= d_X(x, x^*) + d_X(x^*, x') + d_Y(y, y^*) + d_Y(y^*, y') \\
&> d_X(x, x') + d(y, y') = d_{sum}(p, p')
\end{align*}

by the triangle inequality on $d_X$ and $d_Y$. For $d_{max}$, $d_{max}(p, p') = \max \{d_X(x, x'), d_Y(y, y')\}$. Without loss of generality, suppose that $d_X(x, x') \geq d_Y(y, y')$. Then
\begin{align*}
d_{max}(p, p^*) + d_{max}(p^*, p') &= \max \{d_X(x, x^*), d_Y(y, y^*)\} + \max \{d_X(x^*, x'), d_Y(y, y')\} \\
&\geq d_X(x, x^*) + d_X(x^*, x') \geq d_X(x, x') = d_{max}(p, p')
\end{align*}
The argument if $d_X(x, x') < d_Y(y, y')$ is similar. For the triangle inequality on $d_E$,


\subsection*{Problem 40}

Let $M$ be a metric space with metric $d$. Prove that the following are equivalent.
\begin{enumerate}
\item $M$ is homeomorphic to $M$ equipped with the discrete metric.
\item Every function $f: M \rightarrow M$ is continuous.
\item Every bijection $g: M \rightarrow M$ is a homeomorphism.
\item $M$ has no cluster points.
\item Every subset of $M$ is clopen.
\item Every compact subset of $M$ is finite.
\end{enumerate}

If written with no other qualifiers, $M$ is equipped with its metric $d$.

For $1 \rightarrow 2$, 
for any arbitrary set equipped with the discrete metric, if $p_n$ is a sequence such that $p_n \rightarrow p$, then the tail of $p_n = p$. Because $M_{discrete}$ is homeomorphic to $M_d$, this is true on $M_d$. Thus every convergent sequence has a constant tail, which is trivially preserved under an arbitrary function. Thus arbitrary functions are continuous.

Thus for any function $f: M_1 \rightarrow M_2$, sequential limits are trivially preserved, so $f$ is continuous.

For $2 \rightarrow 3$, this follows because $g^{-1}$ exists due to $g$ being a bijection, and $g: M \rightarrow M$ and $g^{-1}: M \rightarrow M$ are both continuous.

For $3 \rightarrow 4$, let $p \in M$ be a cluster point of $M$. Because $p$ is a cluster point, there exists a sequence of distinct points $p_n \in M$ such that $p_n \rightarrow p$. Let $g: M \rightarrow M$ be a function that swaps points $p$ and $p_1$, and is the identity function everywhere else. $g$ is trivially a bijection, and thus it is a homeomorphism, and thus $g$ is continuous and preserves sequential limits. However, $g(p_n) \rightarrow p$, but $g(p) = p_1 \neq p$, contradicting sequential limit preservation. Thus $M$ has no cluster points.

For $4 \rightarrow 5$, because $M$ has no cluster points, for all $p \in M$, there exists some $r > 0$ such that the neighborhood of $M$ about $r$ contains only $p$. Thus the open ball with radius $r$ is contained in $p$, so all singleton sets are open. The arbitrary union of open sets is open, so all subsets of $M$ are open. By taking compliments, all subsets of $M$ are closed, and thus clopen.

For $5 \rightarrow 6$, let $A$ be an infinite, compact subset of $M$, and $A_n$ be an open covering of singletons. Then by compactness, there exists a finite subcover $A_{n_k}$ of $A$. But it's clear that a finite union union of singletons can not cover an infinite set. Thus $A$ can not be infinite.

For $6 \rightarrow 1$, if $d$ is the discrete metric, the proof is trivial. Otherwise, let $d$ be the metric on $M$. I will show that $M$ has no cluster points under metric $d$. From above, that means that all singletons are clopen in $M$ with $d$, thus meaning that all subsets are clopen in $M$ with $d$. Then the identity function between $M$ with $d$ and $M$ with the discrete metric is trivially a homemorphism.

Let $p$ be a cluster point of $M$. Then for every open neighborhood of $p$, there exists a point in $M$ that is not $p$. Consider the open neighborhoods $C_n = B_{r = 1/n}(p)$ for all $n \in \mathbb{Z}^+$. Then for all $n$, there exists $x_n$ such that $x_n \in C_n$.

Consider the set $C = (\cup_{n=1}^{\infty} x_n) \cup p$, and let $A$ be an open cover of $C$. Because $A$ is an open cover, there exists $A_p \in A$ such that $p \in A_p$. Since $A_p$ is open, there exists a neighborhood with $r > 0$ centered at $p$ such that $B_r(p) \subset A_p$. By the construction of $C$, $B_r(p)$ covers all of the $x_n$ with $n > 1/r$. Thus the finite subcover $A_1 \cup A_2 \dots A_p$ covers $A$, making $A$ compact. But this contradicts the assumption that all compact subsets of $M$ are finite. Thus $M$ has no cluster points.

\subsection*{Problem 41}

Let $\| \cdot \|$ be a norm on $\mathbb{R}^m$, and let $B = \{x\in \mathbb{R}^m: \|x\| \leq 1\}$. Prove that $B$ is compact.

Let $\|\cdot \|_E$ be the Euclidean norm on $\mathbb{R}^m$. If $\| \cdot \|$ equals the Euclidean norm, this is trivial.

We first want to show that the identity transformation from $(\mathbb{R}^m, \|\cdot\|_E)$ to $(\mathbb{R}^m, \|\cdot\|)$ is continuous. Due to the translation invariance of norms, if we show that the identity transformation is continuous at the origin, we have shown it for all $\mathbb{R}^m$. We begin with some lemmas. Let $\epsilon > 0$, and $C = \{x \in \mathbb{R}^m: \|x\| = \epsilon\}$.

\begin{lemma}
The identity function from $(\mathbb{R}^m, \|\cdot\|_E)$ to $(\mathbb{R}^m, \|\cdot\|)$ is continuous.
\begin{proof}
All norms on a finite-dimensional vector space are equivalent, which means that open sets under $\|\cdot\|$ are open sets under $\|\cdot\|_E$, and vice versa. Thus the preimage of open sets under the identity transformation is trivially open.
\end{proof}
\end{lemma}

\begin{corollary}
If $A \subset \mathbb{R}^m$ is compact with respect to the Euclidean norm, then it is compact with respect to $\|\cdot\|$.
\begin{proof}
The identity function is continuous, so $A \subset (\mathbb{R}^m, \|\cdot\|)$ is the image of a compact with respect to the Euclidean norm. By the theorems in the book, this implies that $A$ is compact with respect to the $\|\cdot\|$ norm.
\end{proof}
\end{corollary}

Thus, if we can prove that $B$ is compact with respect to the Euclidean norm, it is compact with respect to the norm $\| \cdot \|$. By the Heine-Borel theorem, this is equivalent to $B$ being closed and bounded under the Euclidean norm.

For being closed, norms are continuous functions from $\mathbb{R}^m$ to $\mathbb{R}$. It's clear that $B$ is the preimage of the closed set $[0, 1] \subset \mathbb{R}$, implying that $B$ is closed with respect to $\|\cdot\|$. Since the identity transformation from the Euclidean norm to the $\|\cdot\|$ norm is continuous, this implies that $B$ is closed with respect to the Euclidean norm.

For boundedness, let $A = \{x \in \mathbb{R}^m: \|x\|_E = 1\}$ be the surface of the unit ball in $\mathbb{R}^m$ under the Euclidean norm. $A$ is clearly closed and bounded, so because it is a  subset of $\mathbb{R}^m$, it is a compact. Since all norms are continuous, $\| \cdot \|: \mathbb{R}^m \rightarrow \mathbb{R}$ is a continuous function, and so the image of $A$ under $\| \cdot \|$ obtains maximum and minimum values. Let $y = \min \|A\|$ be the minimum value. Because norms are positive definite, the zero vector is not an element of $A$, so $y > 0$.

Define $B' = \{x \in \mathbb{R}^m : \|x\| \leq y\}$, and $A'$ be the unit ball in $\mathbb{R}^m$. By construction, $B' \subset A'$ and is thus bounded under the Euclidean metric. Dilate $B'$ and $A'$ by a factor of $1/y$. By the norm property that $\|\alpha x \| = |\alpha| \|x\|$, after dilation, $B'$ maps to $B$, and $A'$ maps to a ball with radius 1/x centered at the origin. It's clear that $B$ is a subset of this ball, thus making $B$ bounded under the Euclidean metric.

\subsection*{Problem 43}
Assume that the Cartesian product of two nonempty sets $A \subset M$ and $B \subset M$ is compact in $M \times N$. Prove that $A$ and $B$ are compact.

\begin{proof}
Let $(a_n) \in M$ and $(b_n) \in N$ be arbitrary sequences. Consider their Cartesian product $(a, b)_n \in M \times N$. By compactness of $M \times N$, there exists a convergent subsequence $(a, b)_{n_k} \rightarrow (a, b)$. Since all metrics on a Cartesian product are equivalent, let's use the sum metric, $d_{sum}$. Then for all $\epsilon > 0$ and large $n$, $d_{sum}((a, b)_{n_k}, (a, b)) = d_M(a_{n_k}, a) + d_N(b_{n_k}, b) < \epsilon$. Since distances are nonnegative, this implies that for large $n$ $d_M(a_{n_k}, a) + d_N(b_{n_k}, b) < \epsilon$ and $d_N(b_{n_k}, b) < \epsilon$. Thus $(a_n)$ and $(b_n)$ have convergent subsequences, indicating that $M$ and $N$ are compact.
\end{proof}

\subsection*{Problem 44}

Consider a function $f: M \rightarrow \mathbb{R}$. Its graph is the set

\[
G(f) = \{(p, y) \in M \times \mathbb{R}: y = f(p) \}
\]

(Note the notation is my own invention; I have no idea if it's standard or not).

\subsubsection*{Part a}
Prove that if $f$ is continuous then its graph is closed as a subset of $M \times \mathbb{R}$.

\begin{proof}
By the theorems in the book, the metrics on a Cartesian product are equivalent. I will use $d_{sum}$ for simplicity. Let $d_M$ be the metric on $M$, and $d_E$ be the Euclidean metric on $\mathbb{R}$.

Let $((a, f(a))_n) \in G(f)$ be a convergent sequence such that $(a, f(a))_n \rightarrow (a, b) \in M \times \mathbb{R}$. Because of convergence, this implies that for all $\epsilon > 0$, for large $n$, $d_{sum}((a, f(a))_n, (a, b)) = d_M(a_n, a) + d_E(f(a_n), b) < \epsilon$. Note that this implies that $a_n \rightarrow a$ and $f(a) \rightarrow b$. By continuity, $a_n \rightarrow a$ implies that $f(a_n) \rightarrow f(a)$. Since limits are unique, this implies $f(a) = b$, and so $(a, b) = (a, f(a)) \in M \times \mathbb{R}$. Thus $M \times \mathbb{R}$ is closed.
\end{proof}

\subsubsection*{Part b}
Prove that if $f$ is continuous and $M$ is compact then its graph is compact.

\begin{proof}
Because $f$ is continuous and $M$ is compact, $f(M) \subset \mathbb{R}$ is a compact. Therefore $G(f) = M \times f(M)$ is the Cartesian product of two compacts, and thus compact.
\end{proof}

\subsubsection*{Part c}
Prove that if the graph of $f$ is compact then $f$ is continuous.

If $M$ is empty, then $f$ is trivially continuous. If $M$ is nonempty but $p$ is an isolated point of $M$, then $f$ is trivially continuous at $p$.

Otherwise, let $p$ be a cluster point of $M$ and suppose $f$ is discontinuous at $p$. Then there exists $\epsilon > 0$ such that for all $\delta > 0$, there exists a $x \neq p \in M : d_M(x, p) < \delta$ such that $|f(p) - f(x)| > \epsilon$. 

Therefore for all $n \in \mathbb{N}$, there exists a point $a_n \neq p \in M$ such that $d_M(a_n, p) < \frac{1}{n}$ and $|f(a_n) - f(p)| > \epsilon$.

Consider the sequence $(a_n) \in M$, and its associated sequence $((a_n, f(a_n))) \in M \times \mathbb{R}$. By construction, it's clear that all subsequences of $((a_n, f(a_n)))$ converge to a point with first coordinate $p$. However, because $f$ is a function, the only point in $G(f)$ with first coordinate $p$ is $(p, f(p))$. Also by construction, $f(a_n)$ not converging to $f(p)$ implies that $((a_n, f(a_n)))$ does not converge to $(p, f(p))$. Thus the sequence $((a_n, f(a_n))) \in G(f)$ has no convergent subsequence, which contradicts the assumption that $G(f)$ is compact. Therefore $f$ is continuous at $x$. Since all points in $M$ are either isolated points or cluster points, $f$ is continuous on $M$.

\subsubsection*{Part d}

Give an example of a discontinuous function $f: \mathbb{R} \rightarrow \mathbb{R}$ whose graph is closed.

Consider the function
\[
f(x) = 
\begin{cases}
0 & \text{if } x = 0 \\
\frac{1}{x} & \text{else}
\end{cases}
\]

The singleton at $(0, 0)$ is trivially closed. The two hyperbola arms are closed because they contain all of their limits. Thus, the graph if $f$ is closed. However, the $f$ is obviously discontinuous.

\subsection*{Problem 45}
Draw a Cantor set $C$ on the circle and consider the set $A$ of all chords between points of $C$.

Without loss of generality, I will assume the circle has circumference 1. I will arbitrarily choose a point $p$ to be $0$ and count off distances going counterclockwise.
\subsubsection*{Part a}
Prove that $A$ is compact.

For an interval $I_a$, denote the left and right endpoints of $I_A$ as $A_L$ and $A_R$, respectively.
\begin{lemma}
\label{TwoIntervalChords}
Let $I_A$ and $I_B$ be intervals in $C_n$, the $n$th step of the Cantor set. Without loss of generality, let $A_R < B_L$. Then the set of all chords between the intervals is the shape bounded by the curve $A_L A_R$, the line segment $\overbar{A_R B_L}$, the curve $B_L B_R$, and the line segment $\overbar{A_R B_L}$. Denote this shape $S_{AB}$.
\begin{proof}
All chords must have one of their endpoints on $I_A$ and the other on $I_B$. Since the intervals are disjoint, mental playing around with the picture will show that the curve draws the desired shape.
\end{proof}
\end{lemma}

\begin{lemma}
\label{SubintervalChords}
Let $I_A$ and $I_B$ be intervals as defined in \ref{TwoIntervalChords}, and let $I_C$, $I_D$ be subintervals such that $I_C, I_D \subset I_A \cup I_B$ Then $S_{CD} \subset S_{AB}$.
\begin{proof}
For $p \in S_{CD}$, $p$ is on a chord with one endpoint on $I_C$ and the other on $I_D$. By subintervals, the endpoints are in $I_A \cup I_B$, so the chord that connects those endpoints is contained in $S_{AB}$. Thus $p \in S_{AB}$.
\end{proof}
\end{lemma}

\begin{corollary}
\label{TwoIntervalCompact}
$S_{AB}$ as defined in Lemma \ref{TwoIntervalChords} is compact.
\begin{proof}
The shape is closed and bounded in $\mathbb{R}^2$.
\end{proof}
\end{corollary}

\begin{lemma}
\label{ChordAIsCompact}
Denote $A_n$ the set of all chords that are drawn between intervals in $C_n$. Then $A_n$ is compact.
\begin{proof}
$A_n = \cup_{i, j=1}^n S_{ij}$. In other words, $A_n$ is the union of all $S_{ij}$, where $i, j$ are two intervals in $C_n$. $A_n$ is bounded, and because it is the union of finite closed sets, $A_n$ is closed. Since $A_n \subset \mathbb{R}^2$, $A_n$ is compact.
\end{proof}
\end{lemma}

\begin{lemma}
\label{ChordAIsNested}
$A_{n+1}$ is a subset of $A_n$.
\begin{proof}
Let $p \in A_{n+1}$. Then $p$ is in an $S$ between two intervals of $C_{n+1}$, which are subintervals of an interval or intervals in $C_n$. By Lemma \ref{SubintervalChords}, $p$ is in the corresponding $S$ in $C_n$, so $p \in A_n$.
\end{proof}
\end{lemma}

\begin{lemma}
\label{ChordAIsNonempty}
$A_n$ is nonempty for all $n$.
\begin{proof}
${1/3, 2/3}$ are always points in $A_n$, and their chord is nonempty.
\end{proof}
\end{lemma}

\begin{lemma}
$A$ is compact.
\begin{proof}
$A = \cap_{i=1}^\infty A_i$. By Lemmas \ref{ChordAIsCompact}, \ref{ChordAIsNested}, and \ref{ChordAIsNonempty}, $A$ is the intersection of a nested series of nonempty compacts, and is thus compact. 
\end{proof}
\end{lemma}

\subsection*{Problem 46}

Assume that $A$ and $B$ are compact, disjoint, nonempty subsets of $M$. Prove that there are $a_0 \in A$ and $b_0 \in B$ such that for all $a \in A$ and $b \in B$ we have
\[
d(a_0, b_0) \leq d(a, b)
\]

In other words, $a_0$ and $b_0$ are closest together.

The metric $d: A \times b \rightarrow \mathbb{R}$ is a function from the Cartesian product of $A$ and $B$ to $\mathbb{R}$. As shown in the book, $A$ and $B$ being compact implies $A \times B$ is compact. Furthermore, all metrics are continuous functions. Therefore, $d$ achieves a maximum and minimum value on $\mathbb{R}$. Denote $m$ as the minimum. Because $A \times B$ is nonempty, $m < \infty$, and because $A$ and $B$ are disjoint, $m > 0$. Let $(a_0, b_0)$ be an arbitrary point in the preimage of $m$. Then the desired property follows immediately.

\subsection*{Problem 47}
Suppose $A, B \subset \mathbb{R}^2$.

\subsubsection*{Part a}
If $A$ and $B$ are homeomorphic, are their complements homeomorphic?
No. Let $A = (0, 1)$ and $B$ be the x-axis. $(0, 1)$ is homeomorphic to $\mathbb{R}$, so $A$ is homeomorphic to $B$. However, $A^C$ is connected while $B^C$ is not.

\subsection*{Problem 48}

Prove that there is an embedding of a line as a closed subset of the plane, and an embedding of a line as a bounded subset of the plane, but there is not an embedding of a line as a closed and bounded subset of the plane.

For closed subset, embed the line onto the x-axis on the plane. This is trivially a homeomorphism, and the x-axis is trivially closed.

For the bounded subset, a line is homeomorphic to the interval $(0, 1)$. Embed $(0, 1)$ on the x-axis on the plane. This is trivially bounded.

For the closed and bounded subset, if a line can be embedded onto a closed and bounded subset of the plane, then there exists a homeomorphism $f: \mathbb{R} \rightarrow \mathbb{R}^2$ such that $f(\mathbb{R})$ is closed and bounded. By the Heine-Borel theorem, this implies that $f(\mathbb{R})$ is compact. Since compactness is a topological property, this implies that the line is compact. But this is a contradiction, as $\mathbb{R}$ is not compact.

\subsection*{Problem 52}

Let $(A_n)$ be a nested decreasing sequence of nonempty closed sets in the metric space $M$

\subsubsection*{Part a}

Show that if $M$ is complete and $\text{diam}(A_n) \rightarrow 0$ as $n \rightarrow \infty$, show that $A = \cap A_n$ is exactly one point.

Let $a_n$ be an arbitrary point in $A_n$. Because the $(A_n)$ are decreasing, for $n \geq m$, $A_n \subset A_m$. Because $\text{diam}(A_n) \rightarrow 0$, for all $\epsilon > 0$, there exists an $m \in \mathbb{N}$ such that $\text{diam}(A_m) < \epsilon$. Thus for all $n \geq m$, $a_m, a_n \in A_m$, so $d(a_m, a_n) \leq \text{diam}(A_m) < \epsilon$, implying that $(a_n)$ is Cauchy.

Since $M$ is complete, there exists $a \in M$ such that $a_n \rightarrow a$. Since for all $n$, $a_n \in A_1$ and $A_1$ is closed, $a \in A_1$. Letting $m$ be fixed and considering the tail of $(a_n)$, we see that $a_m \in A_m$ implies that $a \in A_m$, for all $m$. Thus $a \in A$. Because the diameter of a set with more than one distinct points is greater than zero, $a$ is the only point in $A$.

\subsubsection*{Part b}

To what assertions do the sets $[n, \infty)$ provide counterexamples to?

The sets are not closed, and their diameter does not go to zero.

\subsection*{Problem 53}

Suppose that $(K_n)$ is a nested sequence of compact nonempty sets, $K_1 \supset K_2 \supset \dots $, and $K = \cap K_n$. If for some $\mu > 0$, $\diam(K_n) \geq \mu$ for all $n$, is it true that $\diam(K) \geq \mu$?

Yes.

Note that for all $K_n$, there exist $a_n, b_n \in K_n$ such that $d(a_n, b_n) \geq \mu$. To see this, note that for all $n$, the metric is a continuous function $d: K_n \times K_n \rightarrow \R$, and continuous functions achieve a maximum and minimum on a compact set. Since $\diam(K_n)$ is the maximum of $d$ on $K_n \times K_n$, there exists a point $(a_n, b_n) \in K_n \times K_n$ that achieves this maximum.

Repeat this for all $K_n$ to create the sequence $((a_n, b_n))$. Because the sets are nested, $((a_n, b_n)) \in K_1 \times K_1$. Because $K_1$ is compact, $K_1 \times K_1$ is compact, so there exists a subsequence $((a_{n_k}, b_{n_k})) \in K_1 \times K_1$ and a point $(p, q) \in K_1 \times K_1$ such that $((a_{n_k}, b_{n_k})) \rightarrow (p, q)$. $(p, q)$ is also in $K_2 \times K_2$, since all terms of $((a_{n_k}, b_{n_k}))$ except for possibly the first one are in $K_2$. The same is true for $K_3$, and so on. Thus, $(p, q) \in K \times K$.

Now noting that the metric is a continuous function, it preserves sequential limits, so $\lim_{n \rightarrow \infty} d((a_{n_k}, b_{n_k})) = d((p, q))$. Because $d((a_{n_k}, b_{n_k})) \geq \mu$ by definition, $\lim_{n \rightarrow \infty} d((a_{n_k}, b_{n_k})) \geq \mu$. Therefore $d((p, q)) \geq \mu$, and since $(p, q) \in K \times K$, this implies that $\diam(K) \geq \mu$.

\subsection*{Problem 54}

If $f: A \rightarrow B$ and $g: C \rightarrow B$ such that $A \subset C$ and for each $a \in A$ we have $f(a) = g(a)$, then $g$ extends $f$, or $f$ is extended by $g$. Assume that $f : S \rightarrow \R$ is a uniformly continuous function defined on a subset $S$ of a metric space $M$.

\subsubsection*{Part a}

Prove that $f$ extends to a uniformly continuous function $\bar{f}: \bar{S} \rightarrow \R$.

If $S = \bar{S}$, then the result is trivial. Otherwise, from the theorems in the book, we know that $\bar{S} = S \cup S'$, where $S$ is the set of cluster points of $S$. Note that by the definition of cluster point, if $x \in S'$, then there exists $(x_n) \in S$ such that $x_n \rightarrow x$.

We begin with a theorem.

\begin{theorem}
Let $f: S \rightarrow T$ be uniformly continuous between two metric spaces. Then $f$ preserves Cauchy sequences.

\begin{proof}
Let $(x_n) \in S$ be Cauchy. Fix $\epsilon > 0$ and let $\delta > 0$ be the associated distance in uniform continuity. Since $(x_n)$ is Cauchy, there exists $N \in \N$ such that $c, d \geq N$ implies $d_S(x_c, x_d) < \delta$. By uniform continuity, this implies that for all $c, d \geq N$, $d_T(f(x_c), f(x_d)) < \epsilon$, which shows that $f(x_n)$ is Cauchy.
\end{proof}
\end{theorem}

\begin{corollary}
Let $(x_n) \in S$ be a sequence with a limit in $\bar{S}$, that is, considered as a sequence in $\bar{S}$, $x_n \rightarrow x \in \bar{S}$. Let $f: S \rightarrow \R$ be uniformly continuous. Then $\lim_{n \rightarrow \infty} f(x_n)$ exists.

\begin{proof}
If $x \in S$, the proof is trivial due to continuity of $f$. Otherwise, $(x_n)$ is Cauchy because all sequences with limits are Cauchy, and the uniform continuity of $f$ implies that $f(x_n)$ is Cauchy. Since $\R$ is complete, $f(x_n)$ has a limit in $\R$.
\end{proof}
\end{corollary}

\begin{lemma}
\label{FunctionLimitsUnique}
Let $(a_n), (b_n) \in S$ be sequences with the same limit in $\bar{S}$, that is, $a_n, b_n \rightarrow x$. Let $f: S \rightarrow \R$ be uniformly continuous. Then $\lim_{n \rightarrow \infty} f(a_n) = \lim_{n \rightarrow \infty} f(b_n)$.

\begin{proof}
Let $\lim_{n \rightarrow \infty} f(a_n) = L$, $\lim_{n \rightarrow \infty} f(b_n) = M$. Because $a_n$ and $b_n$ considered in $\bar{S}$ are approaching the same point, for large enough $n$, $d(a_n, b_n)$ can be made arbitrarily small. Because the elements of $(a_n), (b_n)$ are in $S$, by the uniform continuity of $f$, $|f(a_n) - f(b_n)|$ can be made arbitrarily small for large $n$. Similarly, because $\lim_{n \rightarrow \infty} f(a_n) = L$ and $\lim_{n \rightarrow \infty} f(b_n) = M$, $|f(a_n)-L|$ and $|f(b_n)-M|$ can be made arbitrarily small for large $n$. By the Triangle Inequality, $|L-M|$ can be made arbitrarily small for large $n$, implying $L = M$.
\end{proof}
\end{lemma}

Now we prove the main result. By the theorems in the book, $x \in S'$ implies that there exists a sequence $(x_n) \in S$ such that $x_n \rightarrow x$. Define $\bar{f}: \bar{S} \rightarrow \R$ as

\[
\bar{f}(x) = 
\begin{cases}
f(x) & x \in S \\
\lim_{n \rightarrow \infty} f(x_n) & \text{ else}
\end{cases}
\]

where $(x_n) \in S, x_n \rightarrow x$ is arbitrary. As stated above, for all $x \in S$, at least one such $(x_n)$ exists, and by Lemma \ref{FunctionLimitsUnique}, all such sequences share the same limit under $f$. Therefore, $\bar{f}$ is uniquely defined. Because $\bar{S} = S \cup S'$, $\bar{f}$ is properly defined. $\bar{f}$ trivially continues $f$.

Now to show uniform continuity. If $p, q \in S$, then the result is trivial. Fix $\epsilon > 0$, and choose $\delta$ for the uniform continuity of $f$ such that for all $a, b \in S$, $d(a, b) < \delta$ implies $|f(b) - f(a)| < \epsilon$.

Let $p, q \in \bar{S}, p, q \in S^C$ such that $d(p, q) \leq \delta/3$. Then $p, q \in S'$. From the above, there exist sequences $(p_n), (q_n) \in S$ such that $p_n \rightarrow p$, $q_n \rightarrow q$. Because $p_n \rightarrow p$, there exists $n_1$ such that for all $n \geq n_1$, $d(p_{n_1}, p) < \delta/3$. Similarly, because $\bar{f}(p_n) \rightarrow \bar{f}(p)$ by construction, there exists $n_2$ such that for all $n \geq n_2$, $|\bar{f}(p_{n_2}) - \bar{f}(p)| \leq \epsilon$. Define $n^* = \max\{n_1, n_2\}$. Define $p' = p_{n^*} \in S$. Define $q' \in S$ similarly.

By construction of $p'$ and $q'$, $d(p', p) < \delta/3$ and $d(q', q) < \delta/3$, and by assumption $d(p, q) < \delta/3$. By the Triangle Inequality, this implies that $d(p', q') < \delta$. Since $p', q' \in S$, the uniform continuity of $f$ implies that $|f(p') - f(q')| = |\bar{f}(p') - \bar{f}(q')| < \epsilon$. By construction of $p'$ and $q'$, $|\bar{f}(p') - \bar{f}(p)| < \epsilon$ and $|\bar{f}(q') - \bar{f}(q)| < \epsilon$. By the Triangle Inequality, this implies that $|\bar{f}(p) - \bar{f}(q)| < 3\epsilon$, which can be made arbitrarily small.

If only one of $p$ or $q$ is in $\bar{S} - S$, the proof is essentially the same. Thus, $p, q \in \bar{S}$ such that $d(p, q) < \delta/3$ implies that $|\bar{f}(p) - \bar{f}(q)| < 3\epsilon$, implying that $\bar{f}$ is uniformly continuous.

\subsubsection*{Part b}

Prove that $\bar{f}$ is the unique continuous extension of $f$ to a function defined on $\bar{S}$.

If $S = \bar{S}$ then uniqueness of the extension is trivial. Otherwise, let $g: \bar{S} \rightarrow \R$ be a continuous extension of $f: S \rightarrow \R$ such that $g \neq \bar{f}$. Then there exists $p \in \bar{S}$ such that $g(p) \neq \bar{f}(p)$. Because $g$ and $\bar{f}$ are both extensions of $f$, for all $x \in S$, $\bar{f}(x) = g(x)$. Thus $p \in \bar{S} - S \subset S'$. Because $p$ is a cluster point of $S$, there exists $(p_n) \in S$ such that $p_n \rightarrow p$.

Because $(p_n) \in S$ and $g$ is an extension of $f$, for all $n$, $g(p_n) = f(p_n)$. Taking limits, $\lim_{n \rightarrow \infty} g(p_n) = \lim_{n \rightarrow \infty} f(p_n) = \bar{f}(p)$, by the definition of $\bar{f}$. By the continuity of $g$, $\lim_{n \rightarrow \infty} g(p_n) = g(p)$. Thus $g(p) = \bar{f}(p)$, contradicting the assumption that $g(p) \neq \bar{f}(p)$. Thus there is no continuous extension $g$ that differs from $\bar{f}$, and $\bar{f}$ is the unique continuous extension of $f$ to $\bar{S}$.

\subsubsection*{Part c}

Prove the same things when $\R$ is replaced with a complete metric space $N$.

The proof is essentially the same. The only properties of $\R$ that we used in the above are that it is a metric space, and that it is complete. Replace all absolute value signs with the metric on $N$, and replace all references to $d$ with references to the metric on $S$.

\subsection*{Problem 55}

The distance from a point $p$ in a metric space $M$ to a nonempty subset $S \subset M$ is defined to be $\dist(p, S) = \inf\{d(p, s): s \in S \}$.

\subsubsection*{Part a}

Show that $p$ is a limit of $S$ if and only if $\dist(p, S) = 0$.

For the forward, $p$ being a limit of $S$ implies that for all $\epsilon > 0$, there exists $s \in S$ such that $d(s, p) < \epsilon$. This is equivalent to the definition of the infimum. For the reverse, $\dist(p, S) = 0$ implies that for all $\epsilon > 0$, there exists $s \in S$ such that $d(s, p) < \epsilon$. Choose $\epsilon_n = 1/n$, and let the associated $s_n \in S$ form a sequence that converges to $p$. Thus $p$ is a limit point of $S$.

\subsubsection*{Part b}

Show that $p \rightarrow \dist(p, S)$ is a uniformly continuous function of $p \in M$.

Let $\delta = \epsilon$, and let $p, q \in M$. We first consider the case when $p$ or $q$ are in $\bar{S}$. Without loss of generality, let $p \in \bar{S} = S \cup S'$. Then $\dist(p, S) = 0$. Since $p \in \bar{S} = S \cup S'$, there exist points in $S$ arbitrarily close to $p$, so $\inf\{d(q, s): s \in S\} \leq d(p, q)$. Thus

\[
|\dist(p, S) - dist(q, S)| = \dist(q, S) = \inf\{d(q, s): s \in S\} \leq d(p, q) < \epsilon
\]

which proves uniform continuity.

If $p, q \in \bar{S}^C$, then we proceed through the equality trichotomy.

\begin{lemma}
Let $p, q \in S$ such that $d(p, q) < \epsilon$. Then $\dist(p, S) \leq \dist(q, S) + \epsilon$
\begin{proof}
Suppose not. Then there exists $\xi > 0$ such that $\dist(p, S) \geq \dist(q, S) + \epsilon + \xi$. Because $\dist(p, S) = \inf\{d(p, s): s \in S\}$ is a greatest lower bound, there exists $s \in S$ such that $\dist(p, S) \leq d(p, s) < \dist(p, S) + \frac{\xi}{2}$. By the Triangle Inequality,

\[
\dist(q, S) \leq d(q, s) \leq d(p, q) + d(p, s) \leq \dist(p, S) + \epsilon + \frac{\xi}{2}
\]

which contradicts the assumption that $\dist(q, S) > \dist(p, S) + \epsilon$.
\end{proof}
\end{lemma}

Thus $\dist(p, S) \leq \dist(q, S) + \epsilon$. By symmetry, this also implies that $\dist(q, S) \leq \dist(p, S) + \epsilon$. Combining the two inequalities gives $|\dist(p, S) - \dist(q, S)| \leq \epsilon$, as desired.

\subsection*{Problem 56}

Show that the 2-sphere is not homeomorphic to the plane.

The 2-sphere is compact, because it is a closed and bounded subset of $\R^3$. If the 2-sphere is homeomorphic to the plane, then because compactness is a topological property, the plane is compact.

I claim that plane is not compact. Cover the plane with open disks with radius 1 centered at all points $\Z^2$. By compactness this should reduce to a finite subcover. However, we can not remove a single disk without losing the covering property. Say we remove the disk centered at the origin. Then there is no other disk that covers the origin, for the disks at $(0, 1), (1, 0)$, etc. are too far away, and the other disks are even further away. Thus this open cover does not reduce to a finite subcover, and the plane is not compact.

\subsection*{Problem 57}

If $S$ is connected, is the interior of $S$ connected? Prove this or give a counterexample.

No. Let $M$ be the $\R^2$ plane, and let $S$ be the union of two disjoint closed disks, with a line drawn between them. $S$ is connected, but the interior of $S$ consists of two disjoint open disks, which are disconnected.

\subsection*{Problem 58}

Theorem 49 in the book states that the closure of a connected set is connected.

\subsubsection*{Part a}

Is the closure of a disconnected set disconnected?

No. Let $M$ be $\R$, and let $S$ be a punctured closed interval. $S$ is disconnected but its closure, a closed interval, is connected.

\subsubsection*{Part b}

What about the interior of a disconnected set?

Still no. Let $M$ be $\R$, and let $S$ be the union of an open interval with a point not contained in that interval. $S$ is disconnected but its interior is just an open interval, which is connected.

\subsection*{Problem 59}

Prove that every countable metric space (not empty and not a singleton) is disconnected.

Connectedness is a topological property. Thus a countable metric space $M$ is either finite, making it homeomorphic to a finite (nonempty and non-singleton) subset of $\R$, or countably infinite, making it homeomorphic to $\R$. Thus this statement is equivalent to proving that $\R$ and finite subsets of $\R$ are disconnected. But this is easy, since all subsets of $\R$ are clopen. All singletons are open in $\R$, and the union of open sets is open, meaning all sets are open, meaning all sets are closed. Since $\R$ or its nonempty or nonsingleton subsets are nonempty, there always exists a partition of $\R$ or its nonempty/nonsingleton subsets into two proper clopen subsets. By homeomorphism, this implies $M$ is disconnected.

\subsection*{Problem 60}

\subsubsection*{Part a}

Prove that a continuous function $f: M \rightarrow \R$, all of whose values are integers is a constant provided that $M$ is connected.

I will assume that $M$ is nonempty. If $M$ is a singleton, the proof is trivial.

Assume that $M$ has at least two elements. Because $f$ is continuous and $M$ is connected, the range of $f$ is connected. Suppose that there are at least two distinct points in the range of $f$. Denote them $x, y \in \Z$. Then the range of $f$ is disconnected. The set $\{x\}$ is proper clopen subset of the range of $f$. It is closed, because singleton sets are always closed in any metric space. It is open, because we can construct an open ball with radius $1/2$ centered at $x$ that is a subset of the set. It is proper, because $y$ is not in $\{x\}$. Thus the range of $f$ is disconnected. Since this is a contradiction, there is only one point in the range of $f$. Thus $f$ is a constant function.

\subsubsection*{Part b}

What if all the values are irrational?

The same conclusion holds, that $f$ is a constant function. Again, if $M$ has only one element, the proof is trivial. Otherwise, let $x, y$ be distinct irrational numbers in the range of $f$. By the denseness of rationals in $\R$, there exists a rational number $p \in (x, y)$.

Consider the metric space $X = \R - \{p\}$, the real line with $p$ removed, and the subset $A = (-\infty, p)$. $A$ and $A^C$ are open in $X$, implying that $A$ is a clopen subset of $X$. Letting $F$ be the range of $f$, because $F \subset X$, we see that $A \cap F$ is a clopen subset of $F$ by the Inheritance Principle. Because $p < y$, $y$ is not in $F$, so $A$ is a proper clopen subset of $F$. Thus the range of $f$ is disconnected, a contradiction. Thus there are no two distinct points $x, y$ in the range of $f$, and $f$ is a constant function. 

\end{document}