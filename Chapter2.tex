\documentclass{article}
\usepackage{amsmath}
\usepackage{amsfonts}
\usepackage{amssymb}

\newenvironment{proof}{\paragraph{Proof:}}{\hfill$\square$}
\newtheorem{theorem}{Theorem}
\newtheorem{lemma}[theorem]{Lemma}
\newtheorem{corollary}[theorem]{Corollary}

\author{Arthur Chen}
\title{Pugh Chapter 2}
\date{\today}

\begin{document}

\section*{Chapter 2 A Taste of Topology}

\subsection*{Problem 6}

Determine whether $d_x(p, q) = \sin |p - q|$ on $[0, \frac{\pi}{2})$ is a metric.

\begin{proof}

It is a metric. Positive definiteness follows from the fact that the absolute value function is a metric over $\mathbb{R}$, and sine being one-to-one over the range of possible functions. Symmetry follows for the same reason. The triangle inequality follows from sine being increasing and concave over $[0, \frac{\pi}{2})$.

Specifically, let $p, r \in [0, \frac{\pi}{2})$, and without loss of generality, let $p \leq r$. If $q = p$ or $q = r$, the triangle inequality is trivial. 

Let $q \notin [p, r]$. If $q > r$, then $q-p > r-p$ implies

\[
d_s(p, q) + d_s(q, r) \geq d_s(p, r) + 0 = \geq d_s(p, r)
\]

A similar result holds if $q < p$. If $q \in (p, r)$, imagine p, q, and r arranged on a line, with p at the origin. As x increases from q to r, the increase in sine is less than the corresponding increase from 0 to r-q, because sine is concave. Thus $\sin(r-p) < \sin(r-q) + \sin(q-p)$.

\end{proof}

\subsection*{Problem 12}

Let $(p_n)$ be a sequence and $f: \mathbb{N} \rightarrow \mathbb{N}$ be a bijection. The sequence $(q_k)_{k\in\mathbb{N}}$ is a rearrangement of $(p_n)$ with $q_k = p_{f(k)}$.

\subsubsection*{Part a}

Are the limits of a sequence unaffected by rearrangement?

The limits of a sequence are unaffected by rearrangement.

Suppose that $(p_n) \rightarrow p$. Then for arbitrary $\epsilon > 0$, there exists an $n \in \mathbb{N}$ such that $i > n$ implies $d(p_i, p) < \epsilon$. This implies that there are at most finite elements of the sequence $(p_n)$ such that $d(p_n, p) \geq \epsilon$. In the rearrangement $(q_n)$, let $M$ be the smallest integer such that $f^{-1}(k) \leq n$. $M$ exists because $\{1, 2, 3\dots n\}$ is finite. Then for all $i > M$, $d(q_i, p) < \epsilon$, and thus $(q_i) \rightarrow p$.

On the other hand, suppose that $(p_n)$ has no limit. Then for arbitrary $p$, there exists an $\epsilon > 0$ such that for all $n \in \mathbb{N}$, there exists $i > n$ such that $d(p_i, p) > \epsilon$. Letting $A(p)$ be the set of these points. $A$ is infinite, because $A$ being finite implies that $A$ has a maximum element.

By contradiction, assume that that $(q_n)$ has a limit of $q$. Then for all $\epsilon_q > 0$, there exists $n(\epsilon_q) \in \mathbb{N}$ such that $i > n(\epsilon_q)$ implies $d(q_i, q) < \epsilon_q$.

Let $\epsilon_q = \epsilon$. Then the set of points $A_q$ such that $d(q_i, q) > \epsilon$ is a subset of $\{1, 2 \dots n(\epsilon)\}$, and is thus finite. But because $(q_n)$ is a rearrangement of $(p_n)$, $A_q = A$. Thus $A_q$ is both finite and infinite, which is a contradiction.

\subsubsection*{Part b}

What if $f$ is an injection?

At first glance, this question seems nonsensical, as for $n\in\mathbb{N}$ which are not in the preimage of $f(\mathbb{N})$, $q_n$ is undefined. If we define such points to have $q_n = 0$, this question becomes intelligible.

Note that if $A$ is an arbitrary set in $\mathbb{N}$, then $f(A)$ has the same cardinality as $A$, because the restriction of $f$ on $A$ is a bijection between the two.

The first result in Part a does not hold. Let $(p_n) = 1$, and let $f(x) = 2x$. The second result still holds, since the cardinality of the sets remains unchanged.

\subsubsection*{Part c}

What if $f$ is a surjection?

The first result in Part a holds. The cardinality of the image of a function must be less than or equal to the cardinality of the domain of the function. Letting $A = \{1, 2 \dots n\}$ and $B = \{n+1, n+2, n+3, \dots\}$, $f(A)$ is finite. Since $f$ is surjective, it covers $\mathbb{N}$, which is infinite. $A + B = \mathbb{N}$ which is the domain of $f$. If $f(B)$ is finite, then $f(A) + f(B) = f(A+B) = f(\mathbb{N})$ is finite, which contradicts the assumption that $f$ is surjective. Therefore $f(B)$ is infinite, and the argument in Part a holds.

The second result in Part b does not hold. Let $(p_n) = \{0, 0, 0, 1, 0, 1, 0, 1 \dots\}$ and let $f(x) = 1$ if $x$ is odd or equals 2, and $\frac{1}{2}x$ if $x$ is even and greater than 3. $f$ is clearly surjective, and $(q_n) = \{0, 1, 1, 1, 1\dots\}$ clearly has a limit of 1.

\subsection*{Problem 13}

Show that if $f: M\rightarrow N$ is a function such that $(p_n)$ converging in $M$ implies that $(f(p_n))$ converges in $N$, then $f$ is continuous.

Note that this is almost the definition of convergence, except for the requirement that $(f(p_n)) \rightarrow f(p)$. Thus showing that $f$ is continuous is equivalent to showing the above requirement.

\begin{proof}

Let $p\in M$ be arbitrary. Let $(a_n)$ be the uniform sequence $a_n = p$ for all $n\in\mathbb{N}$. From the properties of $f$, $(f(a_n))$ converges in $N$, and it converges to $f(p)$. If $(a_n)$ is the only sequence in $M$ that converges to p for all p in $M$ (such as when the discrete metric is used), then $f$ satisfies the sequential convergence condition and is thus continuous.

Let $(b_n)$ be an arbitrary sequence in $M$ such that $b_n \rightarrow p$. Construct the sequence $(c_n)$ such that $c_n = a_{\lceil n/2 \rceil}$ if n is odd and $c_n = b_{n/2}$ if n is even. $(c_n)$ clearly converges to $p$.

By the convergence preservation condition of $f$, $(f(c_n))$ converges in $N$. The subsequence of $(f(c_n))$ consisting of the odd numbers converges to $f(p)$. All subsequences of a convergent sequence converge to the same limit as the main sequence, so $f(c_n) \rightarrow f(p)$, and since $(f(b_n))$ is a subsequence of $(f(c_n))$, $f(b_n) \rightarrow f(p)$. Thus $f$ preserves sequential convergence, and is thus continuous.

\end{proof}

\subsection*{Problem 14}

Let $f: M\rightarrow N$ be a bijection from one metric space to another that preserves distance, i.e. for all $p, q \in M$

\[
d_N(fp, fq) = d_M(p, q)
\]

Then $f$ is called an isometry from $M$ to $N$, and $M$ and $N$ are said to be isometric, $M \equiv N$.

\subsubsection*{Part a}

Prove that every isometry is continuous.

\begin{proof}

Let $(p_n) \in M$ be an arbitrary sequence such that $p_n \rightarrow p \in M$. Then for arbitrary $\epsilon > 0$, there exists an $A \in \mathbb{N}$ such that $a > A$ implies that $d_M(p_a, p) < \epsilon$. Moving to the sequence $(f(p_n))$, by the distance preservation property we have that for the same $\epsilon > 0$, for the same $a > A$, we have $d_N(f(p_a), f(p)) < \epsilon$. Thus $f(p_n) \rightarrow f(p) \in N$, so $f$ preserves sequential convergence and is thus continuous.

\end{proof}

\subsubsection*{Part b}

Prove that every isometry is a homeomorphism.

\begin{proof}

Since $f$ is a bijection, its inverse $f^{-1}$ exists. Since we proved that $f$ is continuous, if we prove that $f^{-1}$ is continuous, then $f$ is a homeomorphism between $M$ and $N$, and thus $M$ and $N$ are homeomorphic.

Let $(p_n) \in N$ be arbitrary such that $p_n \rightarrow p \in N$. Then because $f$ is a bijection, $f^{-1}$, exists, and so $(f^{-1}(p_n))$ and $f^{-1}(p)$ are well defined.

Suppose $f^{-1}$ is not continuous. Then $(f^{-1}(p_n))$ does not converge to $f^{-1}(p)$, and since $f$ is continuous and the inverse of $f^{-1}$, this implies that $(f(f^{-1}(p_n))) = (p_n)$ does not converge to $f(f^{-1}(p)) = p$. But this contradicts the assumption that $p_n \rightarrow p$. Thus $f^{-1}$ is continuous, and $M$ and $N$ are homeomorphic.

\end{proof}

\subsubsection*{Part c}

Prove that $[0, 1]$ is not isometric to $[0, 2]$.

\begin{proof}

Consider the set of pairs of points in $[0, 2]$ that are distance 1 from each other. Because of the total ordering that $[0, 2]$ inherits from $\mathbb{R}$, these pairs can be uniquely identified by their left endpoints. Letting $A$ be the set of left endpoints of such pairs of points, $A$ takes the form $A = \{a: [a, a+1] \subset [0, 2]\} = [0, 1]$.

Let $f$ be an isometry from $[0, 1]$ to $[0, 2]$. By the distance preservation condition of $f$, for all $a \in A$, the preimage of $a$ under $f$ is a point $b \in [0, 1]$ such that the interval $[b, b+1] \subset [0, 1]$. However, the only interval of such form is $[0, 1]$. $f(0)$ can not correspond to $[0, 1]$, as this violates $f$ being a function. Thus an isometry from $[0, 1]$ to $[0, 2]$ does not exist.

\end{proof}

\subsection*{Problem 15}

Prove that isometry is an equivalence relation.

\begin{proof}
Let $M$ be isometric to $N$, and $f: M\rightarrow N$ be the isometry. For arbitrary $p, q \in N$, consider $f^{-1}(p), f^{-1}(p) \in M$. By the distance preserving condition of $f$, $d_M(f^{-1}(p), f^{-1}(q)) = d_N(f(f^{-1}(p)), f(f^{-1}(q))) = d_N(p, q)$. Thus $f^{-1}$ is a bijection that preserves distances from $N$ to $M$, and thus $N$ is isometric to $M$.

Let $M$ be a metric space and $f$ be the identity function. $f$ is clearly a bijection from $M$ to $M$ and distance-preserving, so $f$ is an isometry. Therefore $M$ is isometric to itself.

Let $f:M \rightarrow N$ and $g:N \rightarrow P$ be isometries. Consider their composition $H:M \rightarrow P = (g \circ f)(m)$. $H$ is the composition of bijective functions, and thus is bijective itself, and it's clear that $H$ preserves distances between $M$ and $P$. Thus $M$ is isometric to $P$.
\end{proof}

\subsection*{Problem 16}

Is the perimeter of a square isometric to the circle?  Homeomorphic?

Assuming that the square and the circle are embedded in $\mathbb{R}^2$ with the usual metric, the two are not isometric. If the length diagonal of the square does not equal the diameter of the circle, the proof is immediate. If the length of the diagonal of the square equals the diameter of the circle, call their common distance $d$. There are only two pairs of points on the square with distance $d$ from each other, while there are an infinite number of pairs on the circle distance $d$ from each other. Therefore $f$ can not map between them while remaining a function.

The two are homeomorpic, as one can be stretched into the other.

\subsection*{Problem 18}

Is $\mathbb{R}$ homeomorphic to $\mathbb{Q}$?

No, as the two have different cardinalities, there can not be a bijection between them.

\subsection*{Problem 19}

Is $\mathbb{Q}$ homeomorphic to $\mathbb{N}$?

No. Let $p \in \mathbb{N}$. All convergent sequences $p_n \rightarrow p$ in $\mathbb{N}$ eventually end in repeating $p$. On the other hand, for $q \in \mathbb{Q}$, there exist sequences $(q_n), (r_n) \in \mathbb{Q}$ such that for all $n$, $q_n \neq r_n \neq q$.

Suppose that $f$ is a homeomorphism from $\mathbb{Q}$ to $\mathbb{N}$ such that $f(q) = p \in \mathbb{N}$.  Since $f$ is bicontinuous, it preserves sequential convergence. Thus $(f(q_n))$ and $(f(r_n))$ both converge to $p$. Because of the nature of continuous sequences in $\mathbb{N}$, there exists an $N \in \mathbb{N}$ such that $n > N$ implies that $f(q_n) = f(r_n) = p$. But this implies that $f$ is not injective, and thus not a bijection, which contradicts the assumption that $f$ is a homeomorphism. Thus there exists no homeomorphism between $\mathbb{Q}$ and $\mathbb{N}$.

\subsection*{Problem 20}

What function is a homeomorphism from $(-1, 1)$ to $\mathbb{R}$? Is every open interval homeomorphic to $(0, 1)$?

The function $\tan(\frac{\pi}{2}x)$ is a homeomorphism from $(-1, 1)$ to $\mathbb{R}$. It is bijective, continuous, and its inverse $\frac{2}{\pi}\tan^{-1}(x)$ is continuous.

Every open interval $(a, b)$ is homeomorphic to $(0, 1)$. The function $\frac{x-a}{b-a}$ is bijective and bicontinuous.

\subsection*{Problem 22}

If every closed and bounded subset of a metric space $M$ is compact, does it follow that $M$ is complete?

No. Let $M = {x\in \mathbb{R}: x > 0}$ with the usual metric. The closed and bounded subsets of $M$ are compact, by the same reasoning as in $\mathbb{R}$. However, the Cauchy sequence $(p_n) = \frac{1}{n}$ has no limit in $M$.

\subsection*{Problem 24}

For which intervals $[a, b]$ in $\mathbb{R}$ is the intersection $[a, b]\cup \mathbb{Q}$ a clopen subset of the metric space $\mathbb{Q}$?

The intervals where $a \leq b$ and $a, b$ are irrational numbers. Since $\mathbb{Q}$ is a metric subspace of $\mathbb{R}$, it inherits its open and closed sets from $\mathbb{R}$. Since all intervals of the form $[a, b]$ are closed in $\mathbb{R}$, their intersection with $\mathbb{Q}$ is closed.

These intersections are also open. In $\mathbb{R}$, the interval $(a, b)$ is open, and because $a, b$ are irrational, $a, b \notin \mathbb{Q}$, so $[a, b]\cup\mathbb{Q} = (a, b)\cup\mathbb{Q}$. Thus $[a, b]\cup\mathbb{Q}$ is open.

The above does not hold if $a$ or $b$ is rational. Without any loss of generality, suppose $a$ is rational. Then $a \in [a, b]\cup\mathbb{Q}$, and there is no open ball that contains $a$ and is a subset of  $[a, b]\cup\mathbb{Q}$. Thus $[a, b]\cup\mathbb{Q}$ is not open.

\subsection*{Problem 26}
Prove that a set $U \subset M$ is open if and only if none of its points are the limits of its complement.

For the forward, suppose $x \in U$ is a limit point of $U^C$. Then $x \in \bar{U^C}$, and since for all sets $A \subset \bar{A}$, $x \in U^C$, which contradicts the assumption that $x \in U$. Thus $U$ contains none of the limits of its complement.

For the reverse, suppose that $U$ is not open. Then there exists a $x \in U$  such that for all $r > 0$, the open ball $B_U(x, r)$ is not fully contained in $U$. Thus, for $r = \frac{1}{n}$, we can choose a sequence $x_n \in B_U(x, r)\cup U^C$. $x_n \rightarrow x$ and $x_n \in U^C$ for all $n$, so $x$ is a limit point of $U^C$. This contradicts the assumption that none of $U$'s points are the limit points of its complement, so $U$ is open.

\subsection*{Problem 27}
If $S, T \subset M$, a metric space, and $S \subset T$, prove that
\subsubsection*{Part a}
$\bar{S} \subset \bar{T}$
\begin{proof}
Let $x \in \bar{S}$. Then there exists a sequence $(x_n) \in S$ such that $x_n \rightarrow x$. Since $S \subset T$, $(x_n) \ in T$, so $x_n \rightarrow x$ implies that $x \in \bar{T}$.
\end{proof}

\subsubsection*{Part b}
$\text{int}(S) \subset \text{int}(T)$
\begin{proof}
$\text{int}(S) \subset S \subset T$. Because $\text{int}(S)$ is open and a subset of $T$, it must be contained in the largest open subset of $T$, which is $\text{int}(T)$.
\end{proof}

\subsection*{Problem 28}

A map $f: M \rightarrow N$ is open if $U$ being an open subset of $M$ implies that $f(U)$ is an open subset of $N$.

\subsubsection*{Part a}
If $f$ is open, is it continuous?

No. Let $f$ be the floor function from $\mathbb{R}$ to $\mathbb{Z}$. Every set in $Z$ is open, since singleton sets are open in $Z$ and the union of open sets is open, whether finite or infinite. However, the floor function is not continuous, as the example of $p_n = 1 - \frac{1}{n}$ shows.

\subsubsection*{Part b}
If $f$ is a homeomorphism, is it open?

Yes. Let $f: M \rightarrow N$. Let $A \subset M$ be an arbitrary open set, and $f(A) \subset N$ be its image. Because $f$ is a bijection, it has a continuous inverse function $f^{-1}$. Since $f(A)$ is the preimage of $A$, $f^{-1}(f(A)) = A$ being open implies $f(A)$ is open. Thus open sets are mapped to open sets, and $f$ is open.

\subsubsection*{Part c}
If $f$ is an open, continuous bijection, is it a homeomorphism?

Yes. By similar reasoning to Part b, let $A \subset M$ be an arbitrary open set. Because $f$ is a bijection, $f^{-1}$ exists, and $A$ is the image of some set $f(A) \subset N$. Because $f$ is open, $f(A)$ is open. Thus the preimage of open sets of $f^{-1}$ is an open set, and $f^{-1}$ is continuous. Since $f$ is a bicontinuous bijection, it is a homeomorphism.

\subsubsection*{Part d}
If $f: \mathbb{R} \rightarrow \mathbb{R}$ is a continuous surjection, must it be open?

No. Let $f$ be defined as $f(x) = x \text{ if } x\in\mathbb{Z}$. For $x \in \mathbb{R} - \mathbb{Z}$, define $f$ as

\[
f(x) = 
\begin{cases}
x & x \in \mathbb{Z} \\
\lfloor x \rfloor + 2\text{Remainder}(x) & \text{Remainder}(x) \in (0, \frac{1}{2}) \\
\lceil x \rceil & \text{Remainder}(x) \in [\frac{1}{2}, 1)
\end{cases}
\]

where $\text{Remainder}(x)$ is the non-integer part of $x$. $f(x)$ is essentially a piecewise staircase with a slope of 2 on $[0, \frac{1}{2})$ pieces and a slope of 0 on $(\frac{1}{2}, 1)$ pieces. The function is continuous and surjective, but not open. For example, the open set $(\frac{1}{2}, \frac{3}{4})$ maps to the singleton ${1}$, which is not open.

\subsubsection*{Part e}
If $f: \mathbb{R} \rightarrow \mathbb{R}$ is a continuous, open surjection, must it be a homeomorphism?

Yes. Note that if $f$ is injective, then $f$ is a bijection and Part c implies that it is a homeomorphism.

Suppose that $f$ is not injective. Then there exists $a, b \in \mathbb{R}, a \neq b$ such that $f(a) = f(b)$. Without loss of generality, let $a < b$. Then by continuity, $f$ achieves a maximum value on $[a, b]$.

If $f(a) = M = m$, then $f$ is constant on $[a, b]$. This contradicts the assumption that $f$ is open. For example, the image of the open ball $(a + \frac{b-a}{4}, b - \frac{b-a}{4})$ is a singleton, and thus not open.

If $f(a) < M$, then there exists $c \in (a, b)$ such that $f(c) = M$. Thus there exists $\delta = \min(\frac{c-a}{2}, \frac{b-a}{2})>0$. Consider the open ball $(c-\delta, c+\delta)$ and its image under $f$. Because $f$ is bounded, the image is bounded, and because $f$ is continuous, the image obtains all intermediate values. Thus $f((c-\delta, c+\delta))$ is an interval. Let $x, y$ be the left and right endpoints of the interval, respectively. 

Since $f(c)$ is the maximum of $f$, the right endpoint of the interval is closed. Thus $f((c-\delta, c+\delta))$ has the form $(x, y]$ or $[x, y]$, neither of which is open. Thus, the image of an open set under $f$ is not open, which contradicts the assumption that $f$ is open. A similar argument holds for the minimum. This, $f$ is injective, bijective, and a homeomorphism.

\subsubsection*{Part f}

What happens in Part e if $\mathbb{R}$ is replaced by the unit circle $S^1$?

The result does not hold. Parameterize $S^1$ by the angle $\theta = \arctan(x)$, and let $f: S^1 \rightarrow S^1$ map a point $x$ to the corresponding point at $2\theta$. $f$ is obviously surjective, and it is continuous because it preserves sequential limits. It is open because open sets in $S^1$ consist of open line segments or their unions, and the image of line segments is either another union of line segments, or the entire metric space $S^1$. However, $f$ is not injective. Letting $a$ be the point at $\theta=0$ and $b$ the point at $\theta = \pi$, $a\neq b$, but $f(a) = f(b)$.

\subsection*{Problem 30}

Consider a two-point set $M = \{a, b\}$ whose topology consists of two sets, $M$ and the empty set. Why does this topology not arise from a metric on $M$?

Let $d$ be a metric on $M$. By the properties of metrics, $d(a, b) = d(b, a) = c > 0$, and $d(a, a) = d(b, b) = 0$. The singleton sets ${a}, {b}$ are open, since the open ball $B_d(a, c/2) = {a}$ and $B_d(b, c/2) = {b}$ are contained within themselves. However, the singleton sets are not contained in the topology.  

\subsection*{Problem 31}
Prove the following.
\subsubsection*{Part a}
If $U$ is an open subset of $\mathbb{R}$ then it consists of countably many disjoint intervals $U = \sqcup U_i$. (Unbounded intervals $(-\infty, b)$, $(a, \infty)$, and $(-\infty, \infty)$ are allowed).

\begin{lemma}
The bounded, connected sets on $\mathbb{R}$ are intervals.
\begin{proof}
Let $A$ be bounded and connected. By connectedness, it has the intermediate value property. By boundedness, it has a l.u.b. and g.l.b. Letting $a$ be the g.l.b. and $b$ be the l.u.b. and using the intermediate value property, $(a, b) \subset A$. By the l.u.b., $A \subset [a, b]$. Thus $A$ is an interval. The reverse was shown in the book.
\end{proof}
\end{lemma}

\begin{lemma}
\label{OverlappingIntervalsAreIntervals}
Let $A$ and $B$ be overlapping bounded open intervals. Then $A \cup B$ is an open interval.
\begin{proof}
$A$ and $B$ are connected and share a common point, so $A \cup B$ is connected. Since $A \cup B$ is bounded, $A \cup B$ is an interval. Since $A \cup B$ is open, $A \cup B$ is an open interval.
\end{proof}
\end{lemma}

\begin{corollary}
\label{BoundedIntervalsBecomeDisjoint}
Let ${A_i}$ be a collection of bounded open intervals of $\mathbb{R}$. Then there exists disjoint open intervals ${B_i}$ such that $\cup A_i = \cup B_i$
\begin{proof}
If the $A_i$ are disjoint, the proof is obvious. If not, then there exist $i, j$ such that $A_i$ and $A_j$ overlap. By Lemma \ref{OverlappingIntervalsAreIntervals}, $B_i = A_i \cup A_j$ is an open interval. Substitute $A_i$ and $A_j$ with $B_i$, and repeat while there are still overlapping intervals in $A$.
\end{proof}
\end{corollary}

\begin{lemma}
\label{OpenSetsAreUnionsOfOpenIntervals}
If $A$ is an open subset in $\mathbb{R}$, then there exist bounded open intervals $A_i$ such that $\cup A_i = A$.
\begin{proof}
Because $A$ is open, for all $a \in A$, there exists $r_a > 0$ such that $(a-r_a, a+r_a) \subset A$. Do this for all points in $A$.
\end{proof}
\end{lemma}

\begin{lemma}
Let $\epsilon > 0$. Let $A_i$ be a collection of disjoint open intervals such that $\text{length}(A_i) > \epsilon$ for all $i$. Then $A_i$ is countable.
\begin{proof}
Let $B_i$ be the intervals of $A_i$ dilated by a factor of $1/\epsilon$. That is, if $A_i = (x, y)$, $B_i = (x/\epsilon, y/\epsilon)$. Because all of the $A_i$s have length greater than $1/\epsilon$, the $B_i$ have length greater than 1. Thus the $B_i$ can be uniquely identified with the natural numbers by associating each $B_i$ with the floor of its left endpoint. Since the natural numbers are countable, $B_i$ is at most countable, and there exists a homeomorphism between the $A_i$ and $B_i$, the $A_i$ are countable.
\end{proof}
\end{lemma}

\begin{lemma}
Let $A$ be an open set in $\mathbb{R}$, and let $A = B \sqcup C$. If $B$ is open, then $C$ is open. (Furthermore, $B$ and $C$ are disconnected, but I don't know if I need that right now).
\begin{proof}
TODO
\end{proof}
\end{lemma}

\begin{theorem}
If $A$ is an open subset of $\mathbb{R}$, then it consists of disjoint open intervals $A = \sqcup A_i$ (unbounded open intervals are acceptable).

\begin{proof}
If $A$ is bounded, then by Lemma \ref{OpenSetsAreUnionsOfOpenIntervals}, $A$ can be expressed as the union of open intervals, and by Corollary \ref{BoundedIntervalsBecomeDisjoint}, $A$ can be expressed as the disjoint union of bounded open intervals.

If $A$ is unbounded and $A = \mathbb{R}$, the proof is trivial. Suppose $A$ is unbounded in the positive direction. For simplicity, suppose that $A$ is bounded in the negative direction. Since $A$ does not equal $\mathbb{R}$, $A^C$ is nonempty, and by $A$ being unbounded in the positive direction, $A^C$ is bounded above. Thus $A^C$ has a least upper bound. Denoting the bound as a, $(a, \infty) \subset A$. Because $A$ is open, $A^C$ is closed, and thus $a \in A^C$.

Let $C = A-(a, \infty)$. For all $c \in C$, $c < a$. If $c = a$, then $C \subset A$ implies $c \in A$ contradicts that $a \in A^C$. If $c > a$, then $c \in C$ and $c \in (a, \infty)$ is a contradiction because by construction $C$ and $(a, \infty)$ are disjoint.

USE THE LEMMA TO SHOW THAT THIS IMPLIES THAT C IS OPEN. YOU STILL HAVEN'T SHOWN THAT YET.

Thus $C$ is a bounded open subset of $\mathbb{R}$, and can be expressed as the disjoint union of bounded open intervals $C_i$. Adding back in $(a, \infty)$ gives $A$ as the disjoint union of open intervals $A_i$. The reverse, and the case when $A$ is unbounded in both directions, follow similarly.
\end{proof}
\end{theorem}

\end{document}