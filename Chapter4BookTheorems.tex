\documentclass{amsart}
\usepackage{mathtools}
\DeclarePairedDelimiter{\abs}{\lvert}{\rvert}

\newcommand{\RiemannIntable}{
  \mathfrak{R}
}

\newtheorem{theorem}{Theorem}[subsection]
\newtheorem{lemma}{Lemma}[subsubsection]

\newtheorem{manualtheoreminner}{Theorem}
\newenvironment{manualtheorem}[1]{%
  \renewcommand\themanualtheoreminner{#1}%
  \manualtheoreminner
}{\endmanualtheoreminner}

\author{Arthur Chen}
\title{Pugh Chapter 4}
\date{\today}

\begin{document}

\begin{manualtheorem}{1}
If $f_n$ converges uniformly to $f$ and each $f_n$ is continuous at $x_0$, then $f$ is continuous at $x_0$.
\end{manualtheorem}

\begin{proof}
Let $\frac{\epsilon}{3} > 0$ be arbitrary. Since the $f_n$ are continuous at $x_0$, for all $n$, there exists a $\delta_n >0$ such that for all $x \in (x_0 - \delta_n, x_0 + \delta_n)$,
\[
|f_n(x) - f_n(x_0)| < \frac{\epsilon}{3}
\]

Since the $f_n$ converge uniformly to $f$, for all $x$ in the $f$'s domain, there exists $N$ such that for all $n$ greater than $N$,
\[
|f_n(x) - f(x)| < \frac{\epsilon}{3}
\]

Choose arbitrary $n_0 \geq N$. Since $f_{n_0}$ is continuous, for all $x \in (x_0 - \delta_n, x_0 + \delta_n)$,
\[
|f_{n_0}(x) - f_{n_0}(x_0)| < \frac{\epsilon}{3}
\]

Using the second equation and the Triangle Inequality shows that for all $x \in (x_0 - \delta_n, x_0 + \delta_n)$,
\[
|f(x) - f(x_0)| < \epsilon
\]

Which shows that $f$ is continuous.

\end{proof}

\begin{manualtheorem}{4}
$C_b$ is a complete metric space. That is, every Cauchy sequence converges to a limit in $C_b$.
\end{manualtheorem}

Have:
Let $(p_n)$ be a Cauchy sequence in $C_b$. Then for all $\epsilon > 0$, there exists an $N$ such that for all $m,n \geq N$,
\[
|p_n - p_m| = \sup(|p_n(x) - p_m(x)|: x \in [a, b]) < \epsilon 
\]

Want:
Let $(p_n)$ be a Cauchy sequence in $C_b$. Then there exists $p \in C_b$ such that $(p_n)$ converges to $p$. In other words, for all $\epsilon > 0$, there exists an $N$ such that for all $n \geq N$,
\[
|p_n - p| = \sup(|p_n(x) - p(x)|: x \in [a, b]) < \epsilon 
\]

\end{document}