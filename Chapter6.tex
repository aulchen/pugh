\documentclass{article}
\usepackage{amsmath}
\usepackage{amsfonts}
\usepackage{amssymb}
\usepackage{bbm}


\newenvironment{proof}{\paragraph{Proof:}}{\hfill$\square$}
\newtheorem{theorem}{Theorem}
\newtheorem{lemma}[theorem]{Lemma}
\newtheorem{corollary}[theorem]{Corollary}

\newcommand{\R}{\mathbb{R}}
\newcommand{\Q}{\mathbb{Q}}
\newcommand{\Z}{\mathbb{Z}}
\newcommand{\N}{\mathbb{N}}

\newcommand{\A}{\mathcal{A}}

\author{Arthur Chen}
\title{Chapter 6 Lebesgue Theory}
\date{\today}

\begin{document}
\maketitle

\section*{Problem 1}

\subsection*{Part a}

Show that the definition of linear outer measure is unaffected if we demand that the intervals $I_k$ in the coverings be closed instead of open.

Let $m^*$ be the open linear outer measure (the linear outer measure with open intervals), and $n^*$ be the closed linear outer measure (the same with closed intervals).

We start with a lemma.

\begin{lemma}
A point is a zero set in $m^*$ and $n^*$.
\begin{proof}
The same $(x - \frac{1}{n}, x + \frac{1}{n})$ proof works for both closed and open intervals.
\end{proof}
\end{lemma}

\begin{theorem}
\label{ClosedLinearMeasureOfOpenIntervalIsLength}
The closed linear outer measure of an open interval is its length.
\begin{proof}
Let the open interval be $(a, b)$. Since the closed interval $[a, b]$ covers $(a, b)$ and we take infimums of closed covers, $n^*A \leq b-a$.

To show that $n^*a \geq b-a$, we use induction on $k$, the size of the closed cover. This is essentially the proof of Theorem 2. If we can show that for all closed covers $I_k$ of $(a, b)$, $\sum |I_k| \geq b-a$, the properties of the infimum imply that $n^*A \geq b-a$.

For $k=1$, $[a, b]$ covers $(a, b)$. Moving the left endpoint to the right means that the cover no longer contains $a$, with a similar statement when moving the right endpoint left. Thus all other closed covers of $(a, b)$ have length greater than $b-a$.

Suppose we have shown that for all open intervals $(c, d)$ and closed covers $I_k$ of size $n$, we have $\sum_{k=1}^n |I_k| \geq b-a$. We now have $(a, b)$ which is covered by $J_l$ with size $n+1$. We want to show that $\sum_{l=1}^{n+1} |J_l| \geq (a, b)$.

$a$ is contained in a closed interval, say $J_1$. Thus $a_1 \leq a$. If $b_1 \geq b$, then
\[
\sum_{l=1}^{n+1} |J_l| \geq |J_1| \geq b_1 - a_1 \geq b - a
\]
and the inductive hypothesis is proven. On the other hand, if $b_1 < b$, then the open interval divides into

\[
(a, b) = (a, b_1] \cup (b_1, b)
\]

and $\{J_2 \dots J_{n+1}$ covers $(b_1, b)$. Thus

\[
\sum_{l=1}^{n+1} |J_l| = |J_1| + \sum_{l=2}^{n+1} |J_l| \geq (b_1 - a) + (b-b_1) = b-a
\]
by the inductive hypothesis. Thus the theorem is proved by induction.
\end{proof}
\end{theorem}

\begin{lemma}
The open and closed linear measures of a half-open interval are both its length.
\begin{proof}
Without loss of generality, let $[a, b)$ be the interval. Then there is a point $c \in (a, b)$, and 
\[
[a, b) = [a, c] \cup (c, b)
\]
and by Theorem \ref{ClosedLinearMeasureOfOpenIntervalIsLength} above and Theorem 3 in the book, the closed and open linear outer measures of each sub-interval equal the lengths of the subintervals.
\end{proof}
\end{lemma}

\begin{lemma}
The closed and open linear outer measures of intervals with a side going to infinity are infinite.
\begin{proof}
Suppose not. Then there exists a countable covering of the interval by closed intervals such that the sum of their lengths is finite. This is impossible. The same holds for open linear outer measure. 
\end{proof}
\end{lemma}

\begin{theorem}
For all subsets $A$ of $\R$, $m^*A = n^*A$.
\begin{proof}
All subsets of $\R$ are the disjoint union of countable points and countable intervals. The countable points are zero sets and do not affect the measure. If the intervals have ends going to infinity, the open and closed linear outer measures are both infinity. Otherwise, the disjoint intervals all have equal open and closed linear and outer measure, and thus the sums of those measures are equal.
\end{proof}
\end{theorem}

\subsection*{Part b}

Why does this immediately imply that the middle-thirds Cantor set has linear outer measure zero?

At each stage $C_n$ of the middle-thirds Cantor set, $C_n$ is a subset of $2^n$ closed intervals of length $\frac{1}{3^n}$, with total length $(\frac{2}{3})^n$. This immediately means that the closed linear outer measure, and thus the open linear outer measure, of $C_n$ is $(\frac{2}{3})^n$, which goes to zero.

\subsection*{Part c}

Show that the definition of linear outer measure is unaffected if we drop all openness/closedness requirements on the intervals $I_k$ in the coverings.

The open and closed intervals were addressed in Part a. For intervals of the form $(a, b]$ and $[a, b)$, the intervals are the disjoint union of an open and closed interval.

\subsection*{Part d}

What about planar outer measure? Specifically, what if we demand that the rectangles be squares?

Things remain unchanged. For a rectangle $R$ in the covering of $A$, let $S_1$ be the square with the side length equal to the minimum of the side lengths of the rectangle, and place the square flush against the side of the rectangle. The leftover $R\backslash S_1$ is a rectangle, and denote it $R_2$. Continuing this procedure, we either get termination, or a sequence of squares that converges upwards to the rectangle, maybe with some zero sets on the edges that don't affect the outer measure. Doing this for all rectangles in the covering of $A$ gives a covering of $A$ by squares with the same area.

\section*{Problem 3}

A line in the plane that is parallel to one of the coordinate axes is a planar zero set because it is the Cartesian product of a point (it's a linear zero set) and $\R$.

\subsection*{Part a}

What about a line that is not parallel to the coordinate axis?

A line that is not parallel to a coordinate axis is also a planar zero set. Since the line is not a coordinate axis, it eventually intersects the x-axis with angle $\theta \in (0, \pi/2)$. The rectangles have height $\epsilon \sin(\theta)$, length $\epsilon \cos(\theta)$, and an area of $\epsilon^2 \sin(\theta) \cos(\theta) = \epsilon^2/2 \sin(2\theta)$.

Consider a section of the line of unit length. This section of the line can be covered by $1/\epsilon$ boxes with area $\epsilon^2 \sin(\theta) \cos(\theta)$, with total area $\epsilon \sin(\theta) \cos(\theta)$. Since this can be made arbitrarily small, a non-parallel line segment of unit length has planar measure zero. Since the line that is not parallel to a coordinate axis is a countable union of such segments, it also has planar measure zero.

\subsection*{Part b}

What about the situation in higher dimensions?

It should essentially be analogous. Consider a plane in $\R^3$.  Divide the plane into sections with area one, and cover each section with boxes in which the plane intersects the corners of the boxes, and the intersection of the box and the plane has area $\epsilon > 0$. The volume of the boxes is proportional to $\epsilon^3$, while the total area is proportional to $1/\epsilon^2$. Thus the section of the plane with area one can be covered by boxes with volume proportional to $\epsilon$, and is thus a zero set. Since $\Z^2$ is countable, the entire plane is a zero set.

\section*{Problem 5}

Prove that every closed set in $\R$ or $\R^n$ is a $G_\delta$-set. Does it follow at once that every open set is an $F_\sigma$ set? Why?

Let $X \subset R^n$ be closed. For all $k \in \N$, for every point in $X$, place an open ball with radius $\frac{1}{k}$, take their union, and denote the set $A_k$. By construction, $X \subset A_k$ and $A_k$ is open. Denote $A = \cap_{k=1}^\infty A_k$. By construction, $X \subset A$. $A$ is countable and equals $X$, so $X$ is a $G_\delta$-set. Specifically, let $p \in X^C$. Since $X$ is closed, points of $X$ do not get arbitrarily close to $p$. Thus by the Archimedian property, for some $k \in \N$, $p \in A_k^C$ implies that $p \in A^C$ implies $X^C \subset A^C$ implies $A \subset X$.

It immediately follows by taking compliments that all open sets in $\R^n$ are $F_\delta$ sets. For $Y \subset \R^n$ being open, take the compliment and build $A$, a $G_\delta$ set around $Y^C$. Then $A^C$ is an $F_\sigma$ set for $Y$.

\section*{Problem 6}

Complete the proofs of Theorems 16 and 21 in the unbounded, $n$-dimensional case.

We first begin with a lemma.

\begin{lemma}
The countable union of $F_\sigma$ sets is an $F_\sigma$ set. The countable intersection of $G_\delta$ sets is a $G_\delta$ set.
\begin{proof}
Let $A_i$ be $F_\sigma$ sets, and $A_{i, j}$ be closed sets such that $\cup_{j=1}^\infty A_{i,j} = A_i$. Then

\[
A = \bigcup_{i=1}^\infty = \bigcup_{i=1}^\infty \bigcup_{j=1}^\infty A_{i,j} = \bigcup_{i,j \in \N^2} A_{i,j}
\]

is the countable union of closed sets, making $A$ a $F_\sigma$ set. The proof for $G$ is similar.
\end{proof}
\end{lemma}

Since the numbering has changed, we reproduce the theorem:

\begin{theorem}
Lebesgue measure is \textbf{regular} in the sense that each measurable set $E$ can be sandwiched between an $F_\sigma$ set and a $G_\delta$ set, $F \subset E \subset G$, such that $G \\ F$ is a zero set. Conversely, if there is such an $F \subset E \subset G$ then $E$ is measurable.

\begin{proof}
The converse does not rely on boundedness, so it should still hold in the unbounded case.

For the forward, take $E \subset \R^n$ and denote $H = E^C$. Let $R_i$ be the closed cube in $\R^n$, and place $R_i$s on all points on $\Z^n$. The $R_i$ are measurable and their union equals $\R^n$. The intersections between the cubes have measure zero, and there are countably many intersections, so the intersections between the cubes have no effect on measure.

Use the $R_i$s to split up $E$ and $H_i$ and denote the resulting chunks $E_i$ and $H_i$ - that is, define

\begin{align*}
E_i &= R_i \cap E \\
H_i &= R_i \cap H
\end{align*}

for all $i$. Note that some rectangles $R_i$ intersect both $E$ and $H$.






Adding up all of the chunks, we define

\begin{align*}
U_n &= \bigcup_{i=1}^{\infty} U_{n, i} \\
V_n &= \bigcup_{i=1}^{\infty} V_{n, i}
\end{align*}

The $U_n$ and $V_n$ sets are open and decreasing. We claim that the $U_n$ sets are decreasing towards $E$, because $E_i \subset U_{n, i}$ implies $\cup_{i=1}^\infty E_i = E \subset \cup_{i=1}^\infty U_{n, i} = U_n$.

Define

\[
L_n = V_n^C
\]

The $L_n$ sets are closed and increasing. They are subsets of $E$, and increasing towards $E$. Now to create the $F_\sigma$ and $G_\delta$ sets, define $F = \cap_{n=1}^\infty U_n$ and $G = \cup_{n=1}^\infty L_n$. Thus we have the sandwich, $F \subset E \subset G$.

I claim that

\[
m(L_n) = \sum_{i=1}^{\infty} m(L_{n, i})
\]

By finite additivity, 

\end{proof}
\end{theorem}

\end{document}
