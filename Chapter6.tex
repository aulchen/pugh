\documentclass{article}
\usepackage{amsmath}
\usepackage{amsfonts}
\usepackage{amssymb}
\usepackage{bbm}


\newenvironment{proof}{\paragraph{Proof:}}{\hfill$\square$}
\newtheorem{theorem}{Theorem}
\newtheorem{lemma}[theorem]{Lemma}
\newtheorem{corollary}[theorem]{Corollary}

\newcommand{\R}{\mathbb{R}}
\newcommand{\Q}{\mathbb{Q}}
\newcommand{\Z}{\mathbb{Z}}
\newcommand{\N}{\mathbb{N}}

\newcommand{\A}{\mathcal{A}}

\author{Arthur Chen}
\title{Chapter 6 Lebesgue Theory}
\date{\today}

\begin{document}
\maketitle

\section*{Problem 1}

\subsection*{Part a}

Show that the definition of linear outer measure is unaffected if we demand that the intervals $I_k$ in the coverings be closed instead of open.

Let $m^*$ be the open linear outer measure (the linear outer measure with open intervals), and $n^*$ be the closed linear outer measure (the same with closed intervals).

We start with a lemma.

\begin{lemma}
A point is a zero set in $m^*$ and $n^*$.
\begin{proof}
The same $(x - \frac{1}{n}, x + \frac{1}{n})$ proof works for both closed and open intervals.
\end{proof}
\end{lemma}

\begin{theorem}
\label{ClosedLinearMeasureOfOpenIntervalIsLength}
The closed linear outer measure of an open interval is its length.
\begin{proof}
Let the open interval be $(a, b)$. Since the closed interval $[a, b]$ covers $(a, b)$ and we take infimums of closed covers, $n^*A \leq b-a$.

To show that $n^*a \geq b-a$, we use induction on $k$, the size of the closed cover. This is essentially the proof of Theorem 2. If we can show that for all closed covers $I_k$ of $(a, b)$, $\sum |I_k| \geq b-a$, the properties of the infimum imply that $n^*A \geq b-a$.

For $k=1$, $[a, b]$ covers $(a, b)$. Moving the left endpoint to the right means that the cover no longer contains $a$, with a similar statement when moving the right endpoint left. Thus all other closed covers of $(a, b)$ have length greater than $b-a$.

Suppose we have shown that for all open intervals $(c, d)$ and closed covers $I_k$ of size $n$, we have $\sum_{k=1}^n |I_k| \geq b-a$. We now have $(a, b)$ which is covered by $J_l$ with size $n+1$. We want to show that $\sum_{l=1}^{n+1} |J_l| \geq (a, b)$.

$a$ is contained in a closed interval, say $J_1$. Thus $a_1 \leq a$. If $b_1 \geq b$, then
\[
\sum_{l=1}^{n+1} |J_l| \geq |J_1| \geq b_1 - a_1 \geq b - a
\]
and the inductive hypothesis is proven. On the other hand, if $b_1 < b$, then the open interval divides into

\[
(a, b) = (a, b_1] \cup (b_1, b)
\]

and $\{J_2 \dots J_{n+1}$ covers $(b_1, b)$. Thus

\[
\sum_{l=1}^{n+1} |J_l| = |J_1| + \sum_{l=2}^{n+1} |J_l| \geq (b_1 - a) + (b-b_1) = b-a
\]
by the inductive hypothesis. Thus the theorem is proved by induction.
\end{proof}
\end{theorem}

\begin{lemma}
The open and closed linear measures of a half-open interval are both its length.
\begin{proof}
Without loss of generality, let $[a, b)$ be the interval. Then there is a point $c \in (a, b)$, and 
\[
[a, b) = [a, c] \cup (c, b)
\]
and by Theorem \ref{ClosedLinearMeasureOfOpenIntervalIsLength} above and Theorem 3 in the book, the closed and open linear outer measures of each sub-interval equal the lengths of the subintervals.
\end{proof}
\end{lemma}

\begin{lemma}
The closed and open linear outer measures of intervals with a side going to infinity are infinite.
\begin{proof}
Suppose not. Then there exists a countable covering of the interval by closed intervals such that the sum of their lengths is finite. This is impossible. The same holds for open linear outer measure. 
\end{proof}
\end{lemma}

\begin{theorem}
For all subsets $A$ of $\R$, $m^*A = n^*A$.
\begin{proof}
All subsets of $\R$ are the disjoint union of countable points and countable intervals. The countable points are zero sets and do not affect the measure. If the intervals have ends going to infinity, the open and closed linear outer measures are both infinity. Otherwise, the disjoint intervals all have equal open and closed linear and outer measure, and thus the sums of those measures are equal.
\end{proof}
\end{theorem}

\subsection*{Part b}

Why does this immediately imply that the middle-thirds Cantor set has linear outer measure zero?

At each stage $C_n$ of the middle-thirds Cantor set, $C_n$ is a subset of $2^n$ closed intervals of length $\frac{1}{3^n}$, with total length $(\frac{2}{3})^n$. This immediately means that the closed linear outer measure, and thus the open linear outer measure, of $C_n$ is $(\frac{2}{3})^n$, which goes to zero.

\subsection*{Part c}

Show that the definition of linear outer measure is unaffected if we drop all openness/closedness requirements on the intervals $I_k$ in the coverings.

The open and closed intervals were addressed in Part a. For intervals of the form $(a, b]$ and $[a, b)$, the intervals are the disjoint union of an open and closed interval.

\subsection*{Part d}

What about planar outer measure? Specifically, what if we demand that the rectangles be squares?

Things remain unchanged. For a rectangle $R$ in the covering of $A$, let $S_1$ be the square with the side length equal to the minimum of the side lengths of the rectangle, and place the square flush against the side of the rectangle. The leftover $R\backslash S_1$ is a rectangle, and denote it $R_2$. Continuing this procedure, we either get termination, or a sequence of squares that converges upwards to the rectangle, maybe with some zero sets on the edges that don't affect the outer measure. Doing this for all rectangles in the covering of $A$ gives a covering of $A$ by squares with the same area.

\section*{Problem 3}

A line in the plane that is parallel to one of the coordinate axes is a planar zero set because it is the Cartesian product of a point (it's a linear zero set) and $\R$.

\subsection*{Part a}

What about a line that is not parallel to the coordinate axis?

A line that is not parallel to a coordinate axis is also a planar zero set. Since the line is not a coordinate axis, it eventually intersects the x-axis with angle $\theta \in (0, \pi/2)$. The rectangles have height $\epsilon \sin(\theta)$, length $\epsilon \cos(\theta)$, and an area of $\epsilon^2 \sin(\theta) \cos(\theta) = \epsilon^2/2 \sin(2\theta)$.

Consider a section of the line of unit length. This section of the line can be covered by $1/\epsilon$ boxes with area $\epsilon^2 \sin(\theta) \cos(\theta)$, with total area $\epsilon \sin(\theta) \cos(\theta)$. Since this can be made arbitrarily small, a non-parallel line segment of unit length has planar measure zero. Since the line that is not parallel to a coordinate axis is a countable union of such segments, it also has planar measure zero.

\subsection*{Part b}

What about the situation in higher dimensions?

It should essentially be analogous. Consider a plane in $\R^3$.  Divide the plane into sections with area one, and cover each section with boxes in which the plane intersects the corners of the boxes, and the intersection of the box and the plane has area $\epsilon > 0$. The volume of the boxes is proportional to $\epsilon^3$, while the total area is proportional to $1/\epsilon^2$. Thus the section of the plane with area one can be covered by boxes with volume proportional to $\epsilon$, and is thus a zero set. Since $\Z^2$ is countable, the entire plane is a zero set.

\section*{Problem 5}

Prove that every closed set in $\R$ or $\R^n$ is a $G_\delta$-set. Does it follow at once that every open set is an $F_\sigma$ set? Why?

Let $X \subset R^n$ be closed. For all $k \in \N$, for every point in $X$, place an open ball with radius $\frac{1}{k}$, take their union, and denote the set $A_k$. By construction, $X \subset A_k$ and $A_k$ is open. Denote $A = \cap_{k=1}^\infty A_k$. By construction, $X \subset A$. $A$ is countable and equals $X$, so $X$ is a $G_\delta$-set. Specifically, let $p \in X^C$. Since $X$ is closed, points of $X$ do not get arbitrarily close to $p$. Thus by the Archimedian property, for some $k \in \N$, $p \in A_k^C$ implies that $p \in A^C$ implies $X^C \subset A^C$ implies $A \subset X$.

It immediately follows by taking compliments that all open sets in $\R^n$ are $F_\delta$ sets. For $Y \subset \R^n$ being open, take the compliment and build $A$, a $G_\delta$ set around $Y^C$. Then $A^C$ is an $F_\sigma$ set for $Y$.

\section*{Problem 6}

Complete the proofs of Theorems 16 and 21 in the unbounded, $n$-dimensional case.

We first begin with a lemma.

\begin{lemma}
The countable union of $F_\sigma$ sets is an $F_\sigma$ set. The countable intersection of $G_\delta$ sets is a $G_\delta$ set.
\begin{proof}
Let $A_i$ be $F_\sigma$ sets, and $A_{i, j}$ be closed sets such that $\cup_{j=1}^\infty A_{i,j} = A_i$. Then

\[
A = \bigcup_{i=1}^\infty = \bigcup_{i=1}^\infty \bigcup_{j=1}^\infty A_{i,j} = \bigcup_{i,j \in \N^2} A_{i,j}
\]

is the countable union of closed sets, making $A$ a $F_\sigma$ set. The proof for $G$ is similar.
\end{proof}
\end{lemma}

Since the numbering has changed, we reproduce the theorem:

\begin{theorem}
Lebesgue measure is \textbf{regular} in the sense that each measurable set $E$ can be sandwiched between an $F_\sigma$ set and a $G_\delta$ set, $F \subset E \subset G$, such that $G \backslash F$ is a zero set. Conversely, if there is such an $F \subset E \subset G$ then $E$ is measurable.

\begin{proof}
The converse does not rely on boundedness, so it should still hold in the unbounded case.

For the forward, let $E \subset \R^n$ be arbitrary. Let $R_i$ be the closed cube in $\R^n$, and place $R_i$s on all points on $\Z^n$. The $R_i$ are measurable and their union equals $\R^n$. The intersections between the cubes have measure zero, and there are countably many intersections, so the intersections between the cubes have no effect on measure. Define

\begin{align*}
E_i &= E \cap R_i \\
E_i^C &= R_i \backslash E_i
\end{align*}

for all $R_i$. By construction, $\cup_{i=1}^\infty E_i = E$ and $\cup_{i=1}^\infty E_i^C = E^C$. We will sandwich $E_i$ between sets such that their difference has zero measure, then take their countable union to get the $F$ and $G$ sets for $E$ itself.

There are decreasing open sets $U_{i, n}, V_{i, n}$ with $U_{i, n} \supset E_i$, $V_{i, n} \supset E_i^C$ such that $\lim_{n \rightarrow \infty} m(U_{i,n}) \rightarrow m(E_i)$, $\lim_{n \rightarrow \infty} m(V_{i,n}) \rightarrow m(E_i^C)$ by the definition of outer measure. Since the $E_i$ are measurable, $m(U_{i, n} \backslash E_i) \rightarrow 0$ and $m(V_{i, n} \backslash E_i^C) \rightarrow 0$. Define

\[
K_{i, n} = R_{i,n} \backslash V_{i, n}
\]

The $K_{i,n}$ are closed, increasing subsets of $E_i$ with $m(E_i \backslash K_{i,n}) \rightarrow 0$. As a minor detail, note that $K_{i,n} \cup V_{i,n} \supset R_i$, since $V_{i,n}$ in general contains elements outside of $R_i$.

By the measurability of $R_i$,

\begin{align*}
m(K_{i,n}) &= m(R_i \backslash V_{i,n}) = m(R_{i,n}) - m(V_{i,n} \cap R_i) \\
&= m(R_{i,n}) - m(V_{i,n}) + m(V_{i,n} \cap R_i^C)
\end{align*}

implies as $n \rightarrow \infty$

\[
m(K_i) = m(R_i) - m(V_i) \longrightarrow 
m(K_i) + m(V_i) = 1
\]

by the continuity of measure, since $m(V_{i,n}) \rightarrow m(E_i)$, $E_i \subset R_i$ imply $m(V_{i,n} \cap R_i^C) \rightarrow 0$.

Define
\[
F = \bigcup_{i,n \in \N^2} K_{i, n}
\]

$F$ is an $F_\sigma$ set since it is the countable union of closed sets.

Define
\[
G = \bigcap_{n=1}^\infty \bigcup_{i=1}^\infty U_{i, n}
\]
$G$ is a $G_\sigma$ set, since $\bigcup_{i=1}^\infty U_{i, n}$ is the union of open sets and thus open, and $G$ is the countable intersection of open sets.

We now rewrite $G$ in a useful way for measure.

\begin{lemma}
\[
G = \bigcap_{n=1}^\infty \bigcup_{i=1}^\infty U_{i, n} = \bigcup_{i=1}^\infty \bigcap_{n=1}^\infty U_{i, n}
\]
\begin{proof}
I will denote the left expression $G_1$ and the right expression $G_2$.

According to Wikipedia, the right is always a subset of the left. For the forward, let $n$ be fixed, and let $x \in \bigcup_{i=1}^\infty U_{i, n}$. Then there exists an $i$ that can vary with $n$ such that $x \in U_{i(n), n}$.

Taking the intersection over all $n$, $x \in G_1$ means that for all $n$, we can find an $i(n)$ such that $x \in U_{n, i(n)}$.

The $U_i$ are decreasing in $n$. Thus if $x \in U_{i, n}$ for a given $n$, then for all $m \leq n$, $x \in U_{i, m}$. Using this, we see that for $x \in G_1$, if $x \in U_{i(n), n}$, then for all $m \leq n$, we have $x \in U_{i(n), m}$.

We now show that $G_1 \subset G_2$. For arbitrary $x \in G_1$, for all $n$, there is some $i$ such that $x \in U_{i, n}$. Because of the decreasing sets, this means that there is some $i$ such that for all $n$, $x \in U_{i, n}$. Suppose not. Then there is an $n \in \N$ such that $x \in U_{i, n}$ but $x \notin U_{i, n+1}$. Since $x \in G_1$, there is some other set, call it $j$, such that $x \in U_{j, n+1}$. By the decreasing sets property, this means that for $m \leq n+1$, $x \in U_{j, m}$. Since we can continue this for all $n$, we eventually get a set $U_{i, n}$ that contains $x$ for all $n$.

I'm not totally sure that this argument is complete rigorous, but it's the best I've got. If it's not rigorous, I can probably make it rigorous by somehow showing that there are only a finite number of $U_i$s that can contain $x$.

Thus, $x \in G_1$ means that there is some $i$ such that for all $n \in \N$, $x \in U_{i, n}$. This is precisely the definition of $G_2$.
\end{proof}
\end{lemma}

We now show that $F$ and $G$ sandwich $E$.

\begin{lemma}
$F \subset E \subset G$
\begin{proof}
By construction, for all $i, n \in \N^2$, $E_{i, n} \subset U_{i, n}$ implies for all $n$, $E = \cup_{i=1}^\infty E_{i} \subset \cup_{i=1}^\infty U_{i, n}$ implies $E \subset \cap_{n=1}^\infty \cup_{i=1}^\infty U_{i, n} = G$. Similarly, $V_{i, n} \supset E_i^C = R_i \backslash E_i$ for all $n$ implies $K_{i, n} - R_{i, n} \backslash V_{i, n} \subset E_i$ for all $i, n$ implies $F = \cup_{i,n \in \N^2} K_{i, n} \subset \cup_{i=1}^\infty E_i$.
\end{proof}
\end{lemma}

We now show that $m(G \backslash F) = 0$. For all $i$, $\lim_{n \rightarrow \infty} m(U_{i, n}) = E_i$ and $U_{i, n}$ being decreasing in $n$ and all measures being finite imply

\[
m(\bigcap_{n=1}^\infty U_{i, n}) = m(E_i)
\]

by continuity of measure by above. Similarly, $\lim_{n \rightarrow \infty} m(K_{i, n}) = E_i$ and $K_{i, n}$ being increasing in $n$ imply

\[
m(\bigcup_{n=1}^\infty K_{i, n}) = m(E_i)
\]

Thus by the measurability of $U_{i, n}$ and $U_{i, n} \supset K_{i, n}$,

\[
m(\bigcap_{n=1}^\infty U_{i, n} \backslash \bigcup_{n=1}^\infty K_{i, n})
= m(\bigcap_{n=1}^\infty U_{i, n}) - m(\bigcup_{n=1}^\infty K_{i, n}) = 0
\]

Thus $\bigcap_{n=1}^\infty U_{i, n} \backslash \bigcup_{n=1}^\infty K_{i, n}$ is a zero set. Since taking the countable union over all $i$ gives $G \backslash F$, $G \backslash F$ is a zero set. Thus we have sandwiched $E$ between a $G_\sigma$ set and $F_\delta$ set such that their difference is a zero set, and the theorem is proven.
\end{proof}
\end{theorem}

We now begin the proof of the Measurable Product Theorem in arbitrary spaces, possibly unbounded.

\begin{lemma}
If $A$ or $B$ is a zero set then so is $A \times B$.
\end{lemma}

\begin{proof}
The proof is the same as in the book. Let $A \subset \R^n$, $B \subset \R^k$, and $mA = 0$. Fix $\epsilon > 0$. For all $\ell \in \N$, we can cover $A$ with open boxes $C_i \in \R^n$ whose total $n$-volume is so small that when taking the Cartesian product with boxes $D_\ell$ with coordinates $[-\ell, \ell]$ in $R^k$, we have

\[
m(C_i \times D_\ell) < \frac{\epsilon}{2^\ell}
\]

The union of all these boxes (where increasing $\ell$ causes the open boxes $C_i$ to shrink) covers $A \times \R^k$ and has measure $<\epsilon$. Thus $A \times \R^k$ is a zero set and so is its subset, $A \times B$.
\end{proof}

The proof of Lemma 24 in the book also follows immediately. Now for the Measurable Product Theorem.

\begin{theorem}
If $A \subset \R^n$ and $B \subset \R^k$ are measurable then $A \times B$ is measurable and
\[
m(A \times B) = mA \cdot mB
\]
By convention, $0 \cdot \infty = \infty \cdot 0 = 0$.
\begin{proof}

\end{proof}
\end{theorem}

\end{document}
