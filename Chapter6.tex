\documentclass{article}
\usepackage{amsmath}
\usepackage{amsfonts}
\usepackage{amssymb}
\usepackage{bbm}


\newenvironment{proof}{\paragraph{Proof:}}{\hfill$\square$}
\newtheorem{theorem}{Theorem}
\newtheorem{lemma}[theorem]{Lemma}
\newtheorem{corollary}[theorem]{Corollary}

\newcommand{\R}{\mathbb{R}}
\newcommand{\Q}{\mathbb{Q}}
\newcommand{\Z}{\mathbb{Z}}
\newcommand{\N}{\mathbb{N}}

\author{Arthur Chen}
\title{Chapter 6 Lebesgue Theory}
\date{\today}

\begin{document}
\maketitle

\section*{Problem 1}

\subsection*{Part a}

Show that the definition of linear outer measure is unaffected if we demand that the intervals $I_k$ in the coverings be closed instead of open.

Let $m^*$ be the open linear outer measure (the linear outer measure with open intervals), and $n^*$ be the closed linear outer measure (the same with closed intervals).

We start with a lemma.

\begin{lemma}
A point is a zero set in $m^*$ and $n^*$.
\begin{proof}
The same $(x - \frac{1}{n}, x + \frac{1}{n}$ proof works for both closed and open intervals.
\end{proof}
\end{lemma}

\begin{theorem}
The closed linear outer measure of an open interval is its length.
\begin{proof}
Let the open interval be $(a, b)$. Since the closed interval $[a, b]$ covers $(a, b)$ and we take infimums of closed covers, $n^*A \leq b-a$.

To show that $n^*a \geq b-a$, we use induction on $k$, the size of the closed cover. This is essentially the proof of Theorem 2. If we can show that for all closed covers $I_k$ of $(a, b)$, $\sum |I_k| \geq b-a$, the properties of the infimum imply that $n^*A \geq b-a$.

For $k=1$, $[a, b]$ covers $(a, b)$. Moving the left endpoint to the right means that the cover no longer contains $a$, with a similar statement when moving the right endpoint left. Thus all other closed covers of $(a, b)$ have length greater than $b-a$.

Suppose we have shown that for all open intervals $(c, d)$ and closed covers $I_k$ of size $n$, we have $\sum_{k=1}^n |I_k| \geq b-a$. We now have $(a, b)$ which is covered by $J_l$ with size $n+1$. We want to show that $\sum_{l=1}^{n+1} |J_l| \geq (a, b)$.

$a$ is contained in a closed interval, say $J_1$. Thus $a_1 \leq a$. If $b_1 \geq b$, then
\[
\sum_{l=1}^{n+1} |J_l| \geq |J_1| \geq b_1 - a_1 \geq b - a
\]
and the inductive hypothesis is proven. On the other hand, if $b_1 < b$, then the open interval divides into

\[
(a, b) = (a, b_1] \cup (b_1, b)
\]

and $\{J_2 \dots J_{n+1}$ covers $(b_1, b)$. Thus

\[
\sum_{l=1}^{n+1} |J_l| = |J_1| + \sum_{l=2}^{n+1} |J_l| \geq (b_1 - a) + (b-b_1) = b-a
\]
by the inductive hypothesis. Thus the theorem is proved by induction.
\end{proof}
\end{theorem}

\begin{theorem}
If $A$ is a bounded interval with endpoints $a < b$, then $m^*A = n^*A = b-a$.
\begin{proof}
If the interval is closed, then the closed outer measure is just its length. The linear outer measure was shown to be its length, $b-a$ in Theorem 3.
\end{proof}
\end{theorem}

\section*{Problem 3}

A line in the plane that is parallel to one of the coordinate axes is a planar zero set because it is the Cartesian product of a point (it's a linear zero set) and $\R$.

\subsection*{Part a}

What about a line that is not parallel to the coordinate axis?

A line that is not parallel to a coordinate axis is also a planar zero set. Since the line is not a coordinate axis, it eventually intersects the x-axis with angle $\theta \in (0, \pi/2)$. The rectangles have height $\epsilon \sin(\theta)$, length $\epsilon \cos(\theta)$, and an area of $\epsilon^2 \sin(\theta) \cos(\theta) = \epsilon^2/2 \sin(2\theta)$.

Consider a section of the line of unit length. This section of the line can be covered by $1/\epsilon$ boxes with area $\epsilon^2 \sin(\theta) \cos(\theta)$, with total area $\epsilon \sin(\theta) \cos(\theta)$. Since this can be made arbitrarily small, a non-parallel line segment of unit length has planar measure zero. Since the line that is not parallel to a coordinate axis is a countable union of such segments, it also has planar measure zero.

\subsection*{Part b}

What about the situation in higher dimensions?

It should essentially be analogous. Consider a plane in $\R^3$.  Divide the plane into sections with area one, and cover each section with boxes in which the plane intersects the corners of the boxes, and the intersection of the box and the plane has area $\epsilon > 0$. The volume of the boxes is proportional to $\epsilon^3$, while the total area is proportional to $1/\epsilon^2$. Thus the section of the plane with area one can be covered by boxes with volume proportional to $\epsilon$, and is thus a zero set. Since $\Z^2$ is countable, the entire plane is a zero set.

\end{document}
